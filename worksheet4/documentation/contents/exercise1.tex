\begin{exmp}
    Let \(z_1 = 67.0\). We want to find the normalized binary form of this integer and and ten decimal places accurate. According to lemma \ref{XXX}, we have
    \begin{align*}
        67.0 \div 2 &= 33.0 + 1 \\
        33.0 \div 2 &= 16.0 + 1 \\
        16.0 \div 2 &= 8.0 + 0 \\
        8.0 \div 2 &= 4.0 + 0 \\
        4.0 \div 2 &= 2.0 + 0 \\
        2.0 \div 2 &= 1.0 + 0 \\
        1.0 \div 2 &= 0.0 + 1 \text{,}
    \end{align*}
    therefore, we have \(z_1 = 67.0 = (1000011)_2\). To normalize this number, we just have to move the decimal point six digits to the left. Since \(z_1\) only has seven digits, we do not need to round. We have
    \begin{equation*}
        z_1 = 67.0 = (1.000011 \times 2^6)_2
    \end{equation*}
\end{exmp}
\begin{exmp}
    Let \(z_2 = 287.0\). To find the normalized binary form with respect to ten decimal places, we have
    \begin{align*}
        287.0 \div 2 &= 143.0 + 1 \\
        143.0 \div 2 &= 71.0 + 1 \\
        71.0 \div 2 &= 35.0 + 1 \\
        35.0 \div 2 &= 17.0 + 1 \\
        17.0 \div 2 &= 8.0 + 1 \\
        8.0 \div 2 &= 4.0 + 0 \\
        4.0 \div 2 &= 2.0 + 0 \\
        2.0 \div 2 &= 1.0 + 0 \\
        1.0 \div 2 &= 0.0 + 1 \text{,}
    \end{align*}
    therefore, \(z_2 = 287.0 = (100011111)_2\). Again, there is no need to round any digits. Its normalized binary form is
    \begin{equation*}
        z_2 = 287.0 = (1.00011111 \times 2^8)_2
    \end{equation*}
\end{exmp}
\begin{exmp}
    For a non-integer example, let \(z_3 = 10.625\). To find the binary form of this number, we first separate \(z_3 = 10.0 + 0.625\) and apply the algorithm of \ref{XXX} on each summand. For \(10.0\) we have
    \begin{align*}
        10.0 \div 2 &= 5.0 + 0 \\
        5.0 \div 2 &= 2.0 + 1 \\
        2.0 \div 2 &= 1.0 + 0 \\
        1.0 \div 2 &= 0.0 + 1
    \end{align*}
    and for \(0.625\) we will multiply it with \(2\) until we get \(0\)
    \begin{align*}
        0.625 \times 2 &= 0.25 + 1 \\
        0.25 \times 2 &= 0.5 + 0 \\
        0.5 \times 2 &= 0.0 + 1
    \end{align*}
    Combining both results together, we get \(z_3 = (1010.101)_2\). To normalize, we move the decimal place three digits to the left and we have
    \begin{equation*}
        z_3 = 10.625 = (1.010101 \times 2^3)_2 \text{.}
    \end{equation*}
\end{exmp}
\begin{exmp}
    Perhaps a more interesting example is needed. Let \(z_4 = 1.01\). As we did in \ref{XXX EXAMPLE ABOVE}, we will separate \(z_4\) in two parts; however, we immediately see that \(1\) is \(1\) in both decimal and binary system. We will therefore consider \(0.01\).
    \begin{align*}
        0.01 \times 2 = 0.02 + 0 \\
        0.02 \times 2 = 0.04 + 0 \\
        0.04 \times 2 = 0.08 + 0 \\
        0.08 \times 2 = 0.16 + 0 \\
        0.16 \times 2 = 0.32 + 0 \\
        0.32 \times 2 = 0.64 + 0 \\
        1.28 \times 2 = 0.28 + 1 \\
        0.28 \times 2 = 0.56 + 0 \\
        0.56 \times 2 = 0.12 + 1 \\
        0.12 \times 2 = 0.24 + 0
    \end{align*}
    We could go on, but since we only need to find the normalized binary form with respect to ten decimal places. We have
    \begin{equation*}
        z_4 = 1.01 = (1.0000001010 \times 2^0)_2
    \end{equation*}
    which is already normalized.
\end{exmp}