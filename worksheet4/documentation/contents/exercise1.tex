For all following examples, let the mantissa length be \(t = 8\) and the exponent of the floating point arithmetic be bounded by \(N_{\text{min}} = -5\) and \(N_{\text{max}} = 8\).
%
\begin{exmp} \label{exmp:xmax}
    Given the context as defined above, the largest number that can be represented is \(x_\text{max} = 255\). The calculation is fairly simple; choose the largest exponent possible and fill every digit of the mantissa with ones. In binary, this would be
    \begin{equation*}
        x_\text{max} = (0.11111111)_2 \times 2^8 \text{,}
    \end{equation*}
    or in decimal
    \begin{align*}
        x_\text{max} &= (-1)^{\nu} \cdot 2^N \cdot \sum_{i=1}^{t}x_i \beta^{-i}\\
        &= 2^8 \cdot \left(\frac{1}{2} + \frac{1}{4} + \frac{1}{8} + \frac{1}{16} + \frac{1}{32} + \frac{1}{64} + \frac{1}{128} + \frac{1}{256}\right) \\
        &= 255 \text{.}
    \end{align*}
\end{exmp}
\begin{exmp}
    To find the smallest possible positive value in the defined floating point arithmetic, we proceed similary to example \ref{exmp:xmax}. Set the exponent as small as possible and fill the mantissa with zero but the last place. We have in binary
    \begin{equation*}
        x_\text{min} = (0.00000001)_2 \times 2^{-5}
    \end{equation*}
    which in decimal representation translates to
    \begin{align*}
        x_\text{min} &= (-1)^{\nu} \cdot 2^N \cdot \sum_{i=1}^{t}x_i \beta^{-i} \\
        &= 2^{-5} \times \frac{1}{256} \\
        &= \frac{1}{8192}
    \end{align*}
\end{exmp}

% ===================================================================
% (a)
% ===================================================================



% ===================================================================
% (b)
% ===================================================================

\begin{exmp}
    Let \(z_1 = 67.0\). We want to find the normalized binary form of this integer with ten decimal place accuracy. According to lemma \ref{XXX}, we have
    \begin{align*}
        67.0 \div 2 &= 33.0 + 1 \\
        33.0 \div 2 &= 16.0 + 1 \\
        16.0 \div 2 &= 8.0 + 0 \\
        8.0 \div 2 &= 4.0 + 0 \\
        4.0 \div 2 &= 2.0 + 0 \\
        2.0 \div 2 &= 1.0 + 0 \\
        1.0 \div 2 &= 0.0 + 1 \text{.}
    \end{align*}
    Reading the reminders on the left from bottom to top yields \(z_1 = 67.0 = (1000011)_2\). To normalize this number, we move the decimal point seven digits to the left. Since \(z_1\) only has seven digits, we do not need to cut off any digits. We have
    \begin{equation*}
        z_1 = 67.0 = (0.1000011)_2 \times 2^7
    \end{equation*}
    If one wants to check the validity of the conversion from decimal to binary above, we can check the solution by applying the formula from the other way.
    \begin{equation*}
        (-1)^{\nu} \cdot 2^N \cdot \sum_{i=1}^{t}x_i \beta^{-i} = 2^7 \cdot \left(\frac{1}{2} + \frac{1}{64} + \frac{1}{128}\right) = 128 \cdot \frac{67}{128} = 67
    \end{equation*}
    Now, let's consider the floating point number of \(67.0\). \(N = 7\) is between \(N_{\text{min}} = -5\) and \(N_{\text{max}} = 8\), also \(67.0\) has 7 digits in binary form; therefore, there is no rounding to do which means that \(67.0\) can be represented with the given floating point arithmetic without loss of precision.
    \begin{equation*}
        \text{rd}_8(z_1) = (1.000011)_2 \times 2^6
    \end{equation*}
    Since there is no loss of precision, one can easily conclude that the absolute and relative error of \(67.0\) and \(\text{rd}_8(67.0)\) is zero.
\end{exmp}
\begin{exmp}
    Let \(z_2 = 287.0\). To find the normalized binary form with ten decimal place accuracy, we have
    \begin{align*}
        287.0 \div 2 &= 143.0 + 1 \\
        143.0 \div 2 &= 71.0 + 1 \\
        71.0 \div 2 &= 35.0 + 1 \\
        35.0 \div 2 &= 17.0 + 1 \\
        17.0 \div 2 &= 8.0 + 1 \\
        8.0 \div 2 &= 4.0 + 0 \\
        4.0 \div 2 &= 2.0 + 0 \\
        2.0 \div 2 &= 1.0 + 0 \\
        1.0 \div 2 &= 0.0 + 1 \text{,}
    \end{align*}
    therefore, \(z_2 = 287.0 = (100011111)_2\). Again, there is no need to round any digits. Its normalized binary form is
    \begin{equation*}
        z_2 = 287.0 = (0.100011111)_2 \times 2^9
    \end{equation*}
    In this example, we have an exponent \(N = 9\) which is greater than \(N_{\text{max}} = 8\). This means that with the given floating point arithmetic, we have an overflow and \(287.0\) cannot be rounded to a floating point number. Previously \ref{XXX}, we have shown that \(x_{\text{max}} = 255\) which is another reason why \(z_2 > x_{\text{max}}\) cannot be expressed as a floating point number in this context.
    Most trivially, both absolute and relative error are also undefined for \(287.0\).
\end{exmp}
\begin{exmp}
    For a non-integer example, let \(z_3 = 10.625\). To find the binary form of this number, we first separate \(z_3 = 10.0 + 0.625\) and apply the algorithm of \ref{XXX} on each summand. For \(10.0\) we have
    \begin{align*}
        10.0 \div 2 &= 5.0 + 0 \\
        5.0 \div 2 &= 2.0 + 1 \\
        2.0 \div 2 &= 1.0 + 0 \\
        1.0 \div 2 &= 0.0 + 1
    \end{align*}
    and for \(0.625\) we will multiply it with \(2\) until we get \(0\)
    \begin{align*}
        0.625 \times 2 &= 0.25 + 1 \\
        0.25 \times 2 &= 0.5 + 0 \\
        0.5 \times 2 &= 0.0 + 1
    \end{align*}
    Combining both results together, we get \(z_3 = (1010.101)_2\). To normalize, we move the decimal place three digits to the left and we have
    \begin{equation*}
        z_3 = 10.625 = (1.010101 \times 2^3)_2 \text{.}
    \end{equation*}
\end{exmp}
\begin{exmp}
    Perhaps a more interesting example is needed. Let \(z_4 = 1.01\). As we did in \ref{XXX EXAMPLE ABOVE}, we will separate \(z_4\) in two parts; however, we immediately see that \(1\) is \(1\) in both decimal and binary system. We will therefore consider \(0.01\).
    \begin{align*}
        0.01 \times 2 &= 0.02 + 0 \\
        0.02 \times 2 &= 0.04 + 0 \\
        0.04 \times 2 &= 0.08 + 0 \\
        0.08 \times 2 &= 0.16 + 0 \\
        0.16 \times 2 &= 0.32 + 0 \\
        0.32 \times 2 &= 0.64 + 0 \\
        1.28 \times 2 &= 0.28 + 1 \\
        0.28 \times 2 &= 0.56 + 0 \\
        0.56 \times 2 &= 0.12 + 1 \\
        0.12 \times 2 &= 0.24 + 0
    \end{align*}
    We could go on, but since we only need to find the normalized binary form with respect to ten decimal places. We have
    \begin{equation*}
        z_4 = 1.01 \approx (1.0000001010 \times 2^0)_2
    \end{equation*}
    which is already normalized.
\end{exmp}
\begin{exmp}
    As we already fell into the rabit hole of numbers which have endlessly long binary forms, let's continue with \(z_5 = 0.0002\). For this example, we must stay diligent and iterate many times over the algorithm.
    \begin{align*}
        0.0002 \times 2 &= 0.0004 + 0 \\
        0.0004 \times 2 &= 0.0008 + 0 \\
        0.0008 \times 2 &= 0.0016 + 0 \\
        0.0016 \times 2 &= 0.0032 + 0 \\
        0.0032 \times 2 &= 0.0064 + 0 \\
        0.0064 \times 2 &= 0.0128 + 0 \\
        0.0128 \times 2 &= 0.0256 + 0 \\
        0.0256 \times 2 &= 0.0512 + 0 \\
        0.0512 \times 2 &= 0.1024 + 0 \\
        0.1024 \times 2 &= 0.2048 + 0 \\
        0.2048 \times 2 &= 0.4096 + 0 \\
        0.4096 \times 2 &= 0.8192 + 0 \\
        0.8192 \times 2 &= 0.6384 + 1
    \end{align*}
    We got our first 1! Now we only have to find a maximum of 10 more digits.
    \begin{align*}
        0.6384 \times 2 &= 0.2768 + 1 \\
        0.2768 \times 2 &= 0.5536 + 0 \\
        0.5536 \times 2 &= 0.1072 + 1 \\
        0.1072 \times 2 &= 0.2144 + 0 \\
        0.2144 \times 2 &= 0.4288 + 0 \\
        0.4288 \times 2 &= 0.8576 + 0 \\
        0.8576 \times 2 &= 0.7152 + 1 \\
        0.7152 \times 2 &= 0.4304 + 1 \\
        0.4304 \times 2 &= 0.8608 + 0 \\
        0.8608 \times 2 &= 0.7216 + 1
    \end{align*}
    Therefore, we have \(z_5 = 0.0002 \approx (0.00000000000011010001101)_2\) and normalized we have
    \begin{equation*}
        z_5 = 0.0002 \approx (1.1010001101 \times 2^{-13})_2
    \end{equation*}
\end{exmp}
\begin{exmp}
    For the more mathematically minded, we have last but not least \(z_6 = \frac{1}{3}\).
    \begin{align*}
        \frac{1}{3} \times 2 &= \frac{2}{3} + 0 \\
        \frac{2}{3} \times 2 &= \frac{1}{3} + 1
    \end{align*}
    We already see a patern here; further calculation is not needed. We simply have
    \begin{equation*}
        z_6 = \frac{1}{3} \approx (1.0101010101 \times 2^{-2})_2
    \end{equation*}
\end{exmp}
%
For posterity and stripped from tedious calculation, in the following is a table summerizing the results of \ref{XXX}.
\begin{center}
    \begin{tabular}{| c | c |}
        \hline
        decimal representation & normalized binary representation\\
        \hline
        \(67.0\) & \(1.000011 \times 2^6\) \\
        \(287.0\) & \(1.00011111 \times 2^8\) \\
        \(10.625\) & \(1.010101 \times 2^3\) \\
        \(1.01\) & \(1.0000001010 \times 2^0\) \\
        \(0.0002\) & \(1.1010001101 \times 2^{-13}\) \\
        \(\frac{1}{3}\) & \(1.0101010101 \times 2^{-2}\) \\
        \hline
    \end{tabular}
\end{center}