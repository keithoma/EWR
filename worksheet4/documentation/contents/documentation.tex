\section{Documentation of tools4.py and ab4.py}
\subsection{tools4.py Library API}
\subsubsection{Imports}
\texttt{tools4.py} requires three libraries namely \texttt{decimal}, \texttt{numpy} and \texttt{matplotlib.pyplot}. \texttt{decimal} is used to control the mantissa length for a given floating point number; we import \texttt{numpy} for its float data types; and finally, we use \texttt{matplotlib.pyplot} to draw the plot for the absolute and relative error.
\subsubsection{class Equation}
This is more of an auxiliary class to store the given equation
\begin{equation*}
    \frac{1}{x} - \frac{1}{x + 1} = \frac{1}{x (x + 1)} \text{,}
\end{equation*}
and the two formulas which returns the absolute and the relative error. He can also draw graphs for both of
the errors.
\paragraph{Attributes}
\begin{itemize}
    \item precision\_ (int): the precision set for the whole Equation object; every term inside of an Equation
    object adheres to this precision; note that this attribute should be private and must not be changed
    unless 'change\_precision(self, \_precision)' is called
    \item equation\_context\_ (Context): this is the Context object from the decimal library with its 'prec'
    attribute to 'precision\_' (see above); again this attribute should be private
\end{itemize}
\paragraph{\_\_init\_\_(self, \_precision=28)}
\subparagraph*{Arguments}
\begin{enumerate}
    \item \_precision (int): the precision for the decimal.Decimal object; must not be zero or negative; is directly stored under 'precision\_'; the default value is 28, the same as the default value in the decimal library
\end{enumerate}
\subparagraph*{Returns}
\begin{itemize}
    \item nothing
\end{itemize}
\subparagraph*{Raises}
\begin{itemize}
    \item ValueError: if zero or negative values are passed as '\_precision'
\end{itemize}
\subparagraph*{Description}
She constructs an equation object with respect to the desired precision.
\paragraph{change\_precision(self, \_precision)}
\subparagraph*{Arguments}
\begin{enumerate}
    \item \_precision (int): the precision for the decimal.Decimal object; must not be zero or negative; is
    directly stored under 'precision\_'
\end{enumerate}
\subparagraph*{Returns}
\begin{itemize}
    \item nothing
\end{itemize}
\subparagraph*{Raises}
\begin{itemize}
    \item ValueError: if zero or negative values are passed as '\_precision'
\end{itemize}
\subparagraph*{Description}
Since merely changing the 'precision\_' attribute from the outside won't do anything, this method allows the user to change the precision for a given object by correctly changing the 'prec' attribute of the Context object
\paragraph{left\_side(self, \_x)}
\subparagraph*{Arguments}
\begin{enumerate}
    \item \_x (int): the value for x; 0 and -1 are not allowed and this function will naturally raise a ZeroDivisionError
\end{enumerate}
\subparagraph*{Returns}
\begin{itemize}
    \item (Decimal): the solution for the left side of the equation
\end{itemize}
\subparagraph*{Description}
She represents the left side of the equation
\begin{equation*}
    \frac{1}{x} - \frac{1}{x + 1} \text{.}
\end{equation*}
\paragraph{right\_side(self, \_x)}
\subparagraph*{Arguments}
\begin{enumerate}
    \item \_x (int): the value for x; 0 and -1 are not allowed and this function will naturally raise a ZeroDivisionError
\end{enumerate}
\subparagraph*{Returns}
\begin{itemize}
    \item (Decimal): the solution for the right side of the equation
\end{itemize}
\subparagraph*{Description}
She represents the left side of the equation
\begin{equation*}
    \frac{1}{x (x + 1)} \text{.}
\end{equation*}
\paragraph{absolute\_error(self, \_x)}
\subparagraph*{Arguments}
\begin{enumerate}
    \item \_x (int): the value for x for the equation
\end{enumerate}
\subparagraph*{Returns}
\begin{itemize}
    \item (Decimal): the absolute difference between both side of the equation
\end{itemize}
\subparagraph*{Description}
This methods computes the absolute difference between 'left\_side(self, \_x)' and 'right\_side(self, \_x)'.
\paragraph{relative\_error(self, \_x)}
\subparagraph*{Arguments}
\begin{enumerate}
    \item \_x (int): the value for x for the equation
\end{enumerate}
\subparagraph*{Returns}
\begin{itemize}
    \item (Decimal): the relative difference between both side of the equation
\end{itemize}
\subparagraph*{Description}
This methods computes the relative difference between 'left\_side(self, \_x)' and
\paragraph{draw\_absolute\_error(self, \_x)}
\subparagraph*{Arguments}
\begin{enumerate}
    \item \_x (int): the fixed x for which the graph is drawn
\end{enumerate}
\subparagraph*{Returns}
\begin{itemize}
    \item nothing
\end{itemize}
\subparagraph*{Description}
She draws a two dimensional graph of the absolute error of the equation for a fixed x depending on the mantissa length.
\paragraph{draw\_relative\_error(self, \_x)}
\subparagraph*{Arguments}
\begin{enumerate}
    \item \_x (int): the fixed x for which the graph is drawn
\end{enumerate}
\subparagraph*{Returns}
\begin{itemize}
    \item nothing
\end{itemize}
\subparagraph*{Description}
She draws a two dimensional graph of the relative error of the equation for a fixed x depending on the mantissa length.
\subsubsection{Free Functions}
\paragraph{explore\_machine\_epsilon(float\_type)}
\subparagraph*{Arguments}
\begin{enumerate}
    \item float\_type (class): the class for the float type we want to inspect e.g. np.float32;
\end{enumerate}
\subparagraph*{Returns}
\begin{itemize}
    \item epsilon (float\_type): the machine precision; its data type corresponds to the data type passed as the argument
\end{itemize}
\subparagraph*{Description}
This little algorithm tries to find the machine precision of the given float type, such as np.float16, iterativly. See section \ref{XXX} for the validity of this algorithm.
\subsubsection{main()}
She is our main-function. Use the switch, 'test\_switch', to test various capabilities of this module.
Here, we use as an alternative means to find the machine precision the following formula
\begin{equation*}
    \tau = \frac{7}{3} - \frac{4}{3} - 1
\end{equation*}
for the validity of this formula see section \ref{XXX}.