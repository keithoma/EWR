Consider a hotel with an infinite number of rooms. Each room is currently occupied, but a new guest arrives. So the question is, is it possible to accommodate the newly arrived guest? Intuition would say no, the hotel is fully booked; however, if the receptionist is keen enough, they will come to the conclusion that the solution is surprisingly simple. Every guest in the hotel is moved to the next room i.e. the guest in room one moves to two and the guest in room two moves to three and so on. This will open up the first room for the new guest to take.
Now, this paradox by David Hilbert is bogus in the sense that there are no hotels with infinitely many rooms. Infinity is a concept we as mathematicians often take for granted (even the very first set we've learned about, the set of natural numbers, is infinite), but in the real physical world, there are no such things as infinity. Every amount and every measurement is ultimately finite and even if we try to find infinity in the realm of infinitesimals, we will eventually hit the wall of Planck constant, the smallest physically possible unit of length.
Therefore, funnily enough, the set of the real numbers is nothing more than a misnomer. This is especially troublesome for computers which uses zero and ones to represent data. No amounts of memory are enough for a machine to truly grasp the infinite spirals pi's decimal places generate. Even simple calculations between a large and a small number proved to be challenging for a computer. Not all is lost, however. While our computers do not know the endless ocean of the real numbers, they do know a number system known as the floating point arithmetic. This number system is vastly more limited (finite that is) than the real numbers, but they are useful enough for us to navigate the computer through any calculations. As with every tool, we just have to use them right.
With this goal in mind, we present in this article an introduction into floating point arithmetic. We will first cover the theoretical aspect by giving important lemmas and in the second part, these theories are verified through the output of a Python application.