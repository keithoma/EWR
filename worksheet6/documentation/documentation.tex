% ----------------------------------------------------------------------------------------------------------
% HEADER
% ----------------------------------------------------------------------------------------------------------
%
\documentclass[refman]{scrartcl}
%
\usepackage[english]{babel}
\usepackage{colortbl}
\usepackage{epigraph}
\usepackage{fancyhdr}
\usepackage{graphicx}
\usepackage{hhline}
\usepackage[utf8]{inputenc}
\usepackage[procnames]{listings}
\usepackage{longtable}
\usepackage{tikz}
\usepackage{subcaption}
\usepackage{xcolor}
\usepackage{wrapfig}
\usepackage[nottoc,numbib]{tocbibind}
\usepackage{amsmath}
%
\usepackage[T1]{fontenc}
\usepackage{lmodern}
%
\pagestyle{fancy}
%
\captionsetup[subfigure]{labelformat=empty}
%
\usetikzlibrary{trees}
%
\definecolor{LightCyan}{rgb}{0.60, 1, 1}
\definecolor{LightGreen}{rgb}{0.56, 0.93, 0.56}
\definecolor{LightGrey}{rgb}{0.83, 0.83, 0.83}
\definecolor{Amber}{rgb}{1.0, 0.75, 0.0}
\definecolor{DarkOrange}{rgb}{1.0, 0.55, 0.0}
\definecolor{DarkGreen}{rgb}{0.15, 0.85, 0.3}
%
\definecolor{deepblue}{rgb}{0,0,0.5}
\definecolor{deepred}{rgb}{0.6,0,0}
\definecolor{deepgreen}{rgb}{0,0.5,0}
%
\definecolor{darkmidnightblue}{rgb}{0.0, 0.2, 0.4}
%
\definecolor{keywords}{rgb}{0.54, 0.17, 0.89}
\definecolor{comments}{RGB}{0,0,113}
\definecolor{raed}{RGB}{160,0,0}
\definecolor{green}{RGB}{0,150,0}
 
\lstset{language=Python,
        numbers=left,
        stepnumber=1,
        breakatwhitespace=true,
        numberstyle=\tiny\color{gray},
        basicstyle=\ttfamily\small, 
        keywordstyle=\color{keywords},
        commentstyle=\fontseries{lc}\selectfont\itshape\color{gray},
        stringstyle=\color{raed},
        showstringspaces=false,
        identifierstyle=\color{darkmidnightblue},
        procnamekeys={def,class}}
%
% ----------------------------------------------------------------------------------------------------------
% TITLEPAGE & TABLE OF CONTENTS
% ----------------------------------------------------------------------------------------------------------
%
\begin{document}
\begin{titlepage}
	\centering
	\includegraphics[width=0.15\textwidth]{images/huberlin_logo}\par\vspace{1cm}
	{\scshape\LARGE Humboldt University of Berlin \par}
	\vspace{1cm}
	{\scshape\Large Einf{\"u}hrung in das wissenschaftliche Rechnen \par}
	\vspace{1.5cm}
	{\huge\bfseries XXX\par}
	\vspace{2cm}
	{\Large\itshape Christian Parpart \& Kei Thoma \par}
	\vfill

	\vfill

% Bottom of the page
	{\large \today\par}
\end{titlepage}

\tableofcontents
\newpage
%
% 1 INTRODUCTION
\section{Introduction}
\vspace{2cm}
\epigraph{Hmm, difficult. VERY difficult.}{The Sorting Hat}
Whoever played card games intensively knows that shuffling is not as trivial as some might think at first glance. By nature, many card games require the players to sort the cards e.g. in old maid one sorts for pairs with the same rank, in skat the cards in trick are sorted. Therefore, good shuffling techniques are necessary to avoid clumped up cards from the previous game. However, even more challenging than shuffling is sorting cards back to their initial order. It is already tedious for humans to order playing cards where the correct order is already known. So one can imagine the difficulties computers must have to sort a large set where the final order isn't known. In this paper, we describe and analyze two sorting algorithms, quicksort and heapsort.
%
\section{Description of the Algorithms}
\subsection{Quicksort}
Quicksort is a recursive sorting algorithm. Informally speaking, quicksort first chooses a pivot element (in our case, the last element of the list\footnote{For the purpose of this paper, we define a list to be an ordered set for which an order relation such as \(<\) is defined. In terms of computer science, this equates to an array-like data type (in Python this would be a list) with elements which can be compared. An concrete example would be \((0, 1, 2, 3)\) which is incidentally already sorted according to the smaller relation, \(<\).} is chosen as the pivot\footnote{There are more sophisticated ways to choose the pivot. See section XXX for more information.}), then compares every element of the given list to the pivot placing elements smaller than the pivot to the left and every other element to the right. This partitions the list into two. The initial pivot is placed between the two partitions. Note that the pivot is correctly placed since every element smaller than the pivot are in the left partition. Then, the quicksort algorithm is applied to both partitions. The recursion is broken if the current partition only contains zero or one elements.

More formally, we present the pseudocode for the procedure.

\begin{lstlisting}
def partition(_partition, _low, _high):
    i = _low - 1

    # here, we choose the pivot as the far right element of the
    # partition
    pivot = _partition[_high]

    # from line 12 to line 17 we move every element smaller than 
    # the pivot to the left and every other element to the right
    # then we place the pivot in the middle of the two partitions
    for j in range(_low, _high):
        if _partition[j] < pivot:
            ++i
            _partition[i], _partition[j] = _partition[j], 
                                           _partition[i]
    ++i
    _partition[i], _partition[j] = _partition[j], _partition[i]

    return i

def sort_range(_partition, _low, _high):
    if _low < _high:
        pivot_index = partition(_partition)

        if pivot_index > 0:
            sort_range(_partition, _low, pivot_index - 1)
        sort_range(_partition, pivot_index + 1, _high)


# entry point of the algorithm
def quicksort(_list):
    list_length = len(_list)
    sort_range(_list, 0, list_length - 1)

\end{lstlisting}
%
\section{Heapsort}
\subsection{Intuition and Description}
Heapsort is a sorting algorithm which introduces a data structure, a binary tree (the \textit{heap}). A heap is a nearly complete binary tree. For example, consider the list from previous examples

\begin{equation*}
    (7, 1, 5, 4, 9, 2, 8, 3, 0, 6) \text{.}
\end{equation*}

This list can be arranged into a heap simply by placing the first element of the list at the root, then placing the next two elements as its children and so on (see figure \ref{fig:heap}).
\begin{wrapfigure}{r}{0.5\textwidth}
    \begin{tikzpicture}[level distance=1cm,
        level 1/.style={sibling distance=3cm},
        level 2/.style={sibling distance=1.5cm},
        level 3/.style={sibling distance=1cm}]
        \node {7}
            child {
                node {1}
                child {
                    node {4}
                    child {
                        node {3}
                    }
                    child {
                        node {0}
                    }
                }
                child {
                    node {9}
                    child {
                        node {6}
                    }
                }
            }
            child {
                node {5}
                child {
                    node {2}
                }
                child {
                    node {8}
                }
            };
    \end{tikzpicture}
    \caption{the given list arranged into a heap}\label{fig:heap}
\end{wrapfigure}

In essence, the goal of heapsort is to first sort the heap into a \textit{max-heap} where each parent is larger than each of its children. Then, the root element which by necessity must be the largest element is removed from the heap and the element on the lowest branch (in figure \ref{fig:heap} where 6 currently is) is moved to the root (this procedure as a function is called \textit{build-max-heap}).

It turns out however, that not every recursion needs to check if the heap is a max-heap. After sorting the initial heap, one can assume that the heap is already sorted except for the root. The function which partially sorts the heap after the largest element was removed and the element of the lowest branch is moved to the top is called \textit{heapify}.\cite[p.~135]{bib:introductiontoalgorithms} 

As we did with quicksort, we present the pseudocode for heapsort in the following.

\begin{lstlisting}
def heapsort(_list):
    def heapify(_list, _n, _i):
        largest_element_index = _i

        LEFT_CHILD_INDEX = 2 * _i + 1
        if LEFT_CHILD_INDEX < _n:
            if (_list[LEFT_CHILD_INDEX] > 
                _list[largest_element_index]):
                largest_element_index = LEFT_CHILD_INDEX
    
        RIGHT_CHILD_INDEX = 2 * _i + 2
        IF RIGHT_CHILD_INDEX < _n:
            if (_list[RIGHT_CHILD_INDEX] >
                _list[largest_element_index]):
                largest_element_index = RIGHT_CHILD_INDEX
        
        if largest_element_index != _i:
            _list[_i], _list[largest_element_index] =
              _list[largest_element_index], _list[_i]

            heapify(_list, _n, largest_element_index)
            
    i = floor(len(_list) / 2) - 1

    while i >= 0:
        heapify(_list, len(_list), i)
        --i
    
    i = len(_list) - 1
    while i >= 0:
        _list[0], _list[i] = _list[0], _list[i]
        heapify(_list, i, 0)
        --i
\end{lstlisting}

\subsection{Worked Example}
As before, consider the list

\begin{equation*}
    (7, 1, 5, 6, 9, 2, 8, 3, 0, 6)\text{.}
\end{equation*}

We will sort this list using heapsort in the following (see figure \ref{fig:heapsortexample}). Due to space restrictions, steps between max heaps were condensed into one.

\textbf{Color Key}

\begin{itemize}
    \item the numbers swapped in the last step are in {\color{red}red} and {\color{green}green}, where {\color{green}green} indicates that the number was previously the root of the heap
    \item numbers {\color{gray}grayed} out are the numbers which are correctly sorted (removed from the heap)
\end{itemize}
\subsection{Complexity}

The initial build-max-heap takes time \(O(n)\) and heapify which takes time \(O(\ln(n))\) and is called \(n - 1\) times. Together, this means that the complexity of heapsort is \(O(n \ln(n))\).\cite[p.~136]{bib:introductiontoalgorithms}

Heapsort is not stable. Consider the following list \(2^{*}, 1, 2
)\). If heapsort is applied, this list is sorted to \(2, 1, 2^{*}\) thus the algorithm is unstable.

\begin{center}
%
% 1 to 10
%
\begin{figure}
    \caption{Worked example of heapsort.}\label{fig:heapsortexample}
    \begin{subfigure}[c]{0.5\textwidth}
    \begin{tikzpicture}[level distance=1cm,
        level 1/.style={sibling distance=3cm},
        level 2/.style={sibling distance=1.5cm},
        level 3/.style={sibling distance=1cm}]
        \node {7}
            child {
                node {1}
                child {
                    node {4}
                    child {
                        node {3}
                    }
                    child {
                        node {0}
                    }
                }
                child {
                    node {9}
                    child {
                        node {6}
                    }
                }
            }
            child {
                node {5}
                child {
                    node {2}
                }
                child {
                    node {8}
                }
            };
    \end{tikzpicture}
    \begin{tabular}{ | c | c | c | c | c | c | c | c | c | c | }
        \hline
        7 & 1 & 5 & 4 & 9 & 2 & 8 & 3 & 0 & 6 \\ \hline
    \end{tabular}
    \caption{initial state}
    \end{subfigure}
    \begin{subfigure}[c]{0.5\textwidth}
    \begin{tikzpicture}[level distance=1cm,
        level 1/.style={sibling distance=3cm},
        level 2/.style={sibling distance=1.5cm},
        level 3/.style={sibling distance=1cm}]
        \node {\color{red}1}
            child {
                node {7}
                child {
                    node {4}
                    child {
                        node {3}
                    }
                    child {
                        node {0}
                    }
                }
                child {
                    node {6}
                    child {
                        node {\color{DarkGreen}9}
                    }
                }
            }
            child {
                node {8}
                child {
                    node {2}
                }
                child {
                    node {5}
                }
            };
    \end{tikzpicture}
    \begin{tabular}{ | c | c | c | c | c | c | c | c | c | c | }
        \hline
        \color{red}1 & 7 & 8 & 4 & 6 & 2 & 5 & 3 & 0 & \color{DarkGreen}9 \\ \hline
    \end{tabular}
    \caption{create max heap; then swap {\color{red}1} and {\color{DarkGreen}9}}
    \end{subfigure}
    \begin{subfigure}[c]{0.5\textwidth}
    \begin{tikzpicture}[level distance=1cm,
        level 1/.style={sibling distance=3cm},
        level 2/.style={sibling distance=1.5cm},
        level 3/.style={sibling distance=1cm}]
        \node {\color{red}0}
            child {
                node {7}
                child {
                    node {4}
                    child {
                        node {3}
                    }
                    child {
                        node {\color{DarkGreen}8}
                    }
                }
                child {
                    node {6}
                    child {
                        [LightGrey]
                        node {\color{LightGrey}9}
                    }
                }
            }
            child {
                node {5}
                child {
                    node {2}
                }
                child {
                    node {1}
                }
            };
    \end{tikzpicture}
    \begin{tabular}{ | c | c | c | c | c | c | c | c | c | c | }
        \hline
        \color{red}0 & 7 & 5 & 4 & 6 & 2 & 1 & 3 & \color{DarkGreen}8 & \color{gray}9 \\ \hline
    \end{tabular}
    \caption{create max heap; swap {\color{red}0} and {\color{DarkGreen}8}}
    \end{subfigure}
    \begin{subfigure}[c]{0.5\textwidth}
    \begin{tikzpicture}[level distance=1cm,
        level 1/.style={sibling distance=3cm},
        level 2/.style={sibling distance=1.5cm},
        level 3/.style={sibling distance=1cm}]
        \node {\color{red}3}
            child {
                node {6}
                child {
                    node {4}
                    child {
                        node {\color{DarkGreen}7}
                    }
                    child {
                        [LightGrey]
                        node {\color{LightGrey}8}
                    }
                }
                child {
                    node {0}
                    child {
                        [LightGrey]
                        node {\color{LightGrey}9}
                    }
                }
            }
            child {
                node {5}
                child {
                    node {2}
                }
                child {
                    node {1}
                }
            };
    \end{tikzpicture}
    \begin{tabular}{ | c | c | c | c | c | c | c | c | c | c | }
        \hline
        \color{red}3 & 6 & 5 & 4 & 0 & 2 & 1 & \color{DarkGreen}7 & \color{gray}8 & \color{gray}9 \\ \hline
    \end{tabular}
    \caption{create max heap; swap {\color{red}3} and {\color{DarkGreen}7}}
    \end{subfigure}
    \begin{subfigure}[c]{0.5\textwidth}
    \begin{tikzpicture}[level distance=1cm,
        level 1/.style={sibling distance=3cm},
        level 2/.style={sibling distance=1.5cm},
        level 3/.style={sibling distance=1cm}]
        \node {\color{red}1}
            child {
                node {4}
                child {
                    node {3}
                    child {
                        [LightGrey]
                        node {\color{LightGrey}7}
                    }
                    child {
                        [LightGrey]
                        node {\color{LightGrey}8}
                    }
                }
                child {
                    node {0}
                    child {
                        [LightGrey]
                        node {\color{LightGrey}9}
                    }
                }
            }
            child {
                node {5}
                child {
                    node {2}
                }
                child {
                    node {\color{DarkGreen}6}
                }
            };
    \end{tikzpicture}
    \begin{tabular}{ | c | c | c | c | c | c | c | c | c | c | }
        \hline
        \color{red}1 & 4 & 5 & 3 & 0 & 2 & \color{DarkGreen}6 & \color{gray}7 & \color{gray}8 & \color{gray}9 \\ \hline
    \end{tabular}
    \caption{create max heap; swap {\color{red}1} and {\color{DarkGreen}6}}
    \end{subfigure}
    \begin{subfigure}[c]{0.5\textwidth}
    \begin{tikzpicture}[level distance=1cm,
        level 1/.style={sibling distance=3cm},
        level 2/.style={sibling distance=1.5cm},
        level 3/.style={sibling distance=1cm}]
        \node {\color{red}1}
            child {
                node {4}
                child {
                    node {3}
                    child {
                        [LightGrey]
                        node {\color{LightGrey}7}
                    }
                    child {
                        [LightGrey]
                        node {\color{LightGrey}8}
                    }
                }
                child {
                    node {0}
                    child {
                        [LightGrey]
                        node {\color{LightGrey}9}
                    }
                }
            }
            child {
                node {2}
                child {
                    node {\color{DarkGreen}5}
                }
                child {
                    [LightGrey]
                    node {\color{LightGrey}6}
                }
            };
    \end{tikzpicture}
    \begin{tabular}{ | c | c | c | c | c | c | c | c | c | c | }
        \hline
        \color{red}1 & 4 & 2 & 3 & 0 & \color{DarkGreen}5 & \color{gray}6 & \color{gray}7 & \color{gray}8 & \color{gray}9 \\ \hline
    \end{tabular}
    \caption{create max heap; swap {\color{red}1} and {\color{DarkGreen}5}}
    \end{subfigure}
        \begin{subfigure}[c]{0.5\textwidth}
        \begin{tikzpicture}[level distance=1cm,
            level 1/.style={sibling distance=3cm},
            level 2/.style={sibling distance=1.5cm},
            level 3/.style={sibling distance=1cm}]
            \node {\color{red}0}
                child {
                    node {3}
                    child {
                        node {1}
                        child {
                            [LightGrey]
                            node {\color{LightGrey}7}
                        }
                        child {
                            [LightGrey]
                            node {\color{LightGrey}8}
                        }
                    }
                    child {
                        node {\color{DarkGreen}4}
                        child {
                            [LightGrey]
                            node {\color{LightGrey}9}
                        }
                    }
                }
                child {
                    node {2}
                    child {
                        [LightGrey]
                        node {\color{LightGrey}5}
                    }
                    child {
                        [LightGrey]
                        node {\color{LightGrey}6}
                    }
                };
    \end{tikzpicture}
    \begin{tabular}{ | c | c | c | c | c | c | c | c | c | c | }
        \hline
        \color{red}0 & 3 & 2 & 1 & \color{DarkGreen}4 & \color{gray}5 & \color{gray}6 & \color{gray}7 & \color{gray}8 & \color{gray}9 \\ \hline
    \end{tabular}
    \caption{create max heap; swap {\color{red}0} and {\color{DarkGreen}4}}
    \end{subfigure}
    \begin{subfigure}[c]{0.5\textwidth}
        \begin{tikzpicture}[level distance=1cm,
            level 1/.style={sibling distance=3cm},
            level 2/.style={sibling distance=1.5cm},
            level 3/.style={sibling distance=1cm}]
            \node {\color{red}0}
                child {
                    node {1}
                    child {
                        node {\color{DarkGreen}3}
                        child {
                            [LightGrey]
                            node {\color{LightGrey}7}
                        }
                        child {
                            [LightGrey]
                            node {\color{LightGrey}8}
                        }
                    }
                    child {
                        [LightGrey]
                        node {\color{LightGrey}4}
                        child {
                            [LightGrey]
                            node {\color{LightGrey}9}
                        }
                    }
                }
                child {
                    node {2}
                    child {
                        [LightGrey]
                        node {\color{LightGrey}5}
                    }
                    child {
                        [LightGrey]
                        node {\color{LightGrey}6}
                    }
                };
    \end{tikzpicture}
    \begin{tabular}{ | c | c | c | c | c | c | c | c | c | c | }
        \hline
        \color{red}0 & 1 & 2 & \color{DarkGreen}3 & \color{gray}4 & \color{gray}5 & \color{gray}6 & \color{gray}7 & \color{gray}8 & \color{gray}9 \\ \hline
    \end{tabular}
    \caption{create max heap; swap {\color{red}0} and {\color{DarkGreen}3}}
    \end{subfigure}
\end{figure}
\begin{figure}
    \begin{subfigure}[c]{0.5\textwidth}
        \begin{tikzpicture}[level distance=1cm,
            level 1/.style={sibling distance=3cm},
            level 2/.style={sibling distance=1.5cm},
            level 3/.style={sibling distance=1cm}]
            \node {\color{red}0}
                child {
                    node {1}
                    child {
                        [LightGrey]
                        node {\color{LightGrey}3}
                        child {
                            [LightGrey]
                            node {\color{LightGrey}7}
                        }
                        child {
                            [LightGrey]
                            node {\color{LightGrey}8}
                        }
                    }
                    child {
                        [LightGrey]
                        node {\color{LightGrey}4}
                        child {
                            [LightGrey]
                            node {\color{LightGrey}9}
                        }
                    }
                }
                child {
                    node {\color{DarkGreen}2}
                    child {
                        [LightGrey]
                        node {\color{LightGrey}5}
                    }
                    child {
                        [LightGrey]
                        node {\color{LightGrey}6}
                    }
                };
    \end{tikzpicture}
    \begin{tabular}{ | c | c | c | c | c | c | c | c | c | c | }
        \hline
        \color{red}0 & 1 & \color{DarkGreen}2 & \color{gray}3 & \color{gray}4 & \color{gray}5 & \color{gray}6 & \color{gray}7 & \color{gray}8 & \color{gray}9 \\ \hline
    \end{tabular}
    \caption{create max heap; swap {\color{red}0} and {\color{DarkGreen}2}}
    \end{subfigure}
    \begin{subfigure}[c]{0.5\textwidth}
        \begin{tikzpicture}[level distance=1cm,
            level 1/.style={sibling distance=3cm},
            level 2/.style={sibling distance=1.5cm},
            level 3/.style={sibling distance=1cm}]
            \node {\color{red}0}
                child {
                    node {\color{DarkGreen}1}
                    child {
                        [LightGrey]
                        node {\color{LightGrey}3}
                        child {
                            [LightGrey]
                            node {\color{LightGrey}7}
                        }
                        child {
                            [LightGrey]
                            node {\color{LightGrey}8}
                        }
                    }
                    child {
                        [LightGrey]
                        node {\color{LightGrey}4}
                        child {
                            [LightGrey]
                            node {\color{LightGrey}9}
                        }
                    }
                }
                child {
                    [LightGrey]
                    node {\color{LightGrey}2}
                    child {
                        [LightGrey]
                        node {\color{LightGrey}5}
                    }
                    child {
                        [LightGrey]
                        node {\color{LightGrey}6}
                    }
                };
    \end{tikzpicture}
    \begin{tabular}{ | c | c | c | c | c | c | c | c | c | c | }
        \hline
        \color{red}0 & \color{DarkGreen}1 & \color{gray}2 & \color{gray}3 & \color{gray}4 & \color{gray}5 & \color{gray}6 & \color{gray}7 & \color{gray}8 & \color{gray}9 \\ \hline
    \end{tabular}
    \caption{create max heap; swap {\color{red}0} and {\color{DarkGreen}1}}
    \end{subfigure}
    \begin{subfigure}[c]{0.5\textwidth}
        \begin{tikzpicture}[level distance=1cm,
            level 1/.style={sibling distance=3cm},
            level 2/.style={sibling distance=1.5cm},
            level 3/.style={sibling distance=1cm}]
            \node {\color{DarkGreen}0}
                child {
                    [LightGrey]
                    node {\color{LightGrey}1}
                    child {
                        [LightGrey]
                        node {\color{LightGrey}3}
                        child {
                            [LightGrey]
                            node {\color{LightGrey}7}
                        }
                        child {
                            [LightGrey]
                            node {\color{LightGrey}8}
                        }
                    }
                    child {
                        [LightGrey]
                        node {\color{LightGrey}4}
                        child {
                            [LightGrey]
                            node {\color{LightGrey}9}
                        }
                    }
                }
                child {
                    [LightGrey]
                    node {\color{LightGrey}2}
                    child {
                        [LightGrey]
                        node {\color{LightGrey}5}
                    }
                    child {
                        [LightGrey]
                        node {\color{LightGrey}6}
                    }
                };
    \end{tikzpicture}
    \begin{tabular}{ | c | c | c | c | c | c | c | c | c | c | }
        \hline
        \color{DarkGreen}0 & \color{gray}1 & \color{gray}2 & \color{gray}3 & \color{gray}4 & \color{gray}5 & \color{gray}6 & \color{gray}7 & \color{gray}8 & \color{gray}9 \\ \hline
    \end{tabular}
    \caption{create max heap; swap {\color{DarkGreen}0} and {\color{DarkGreen}0}}
    \end{subfigure}
    \begin{subfigure}[c]{0.5\textwidth}
        \begin{tikzpicture}[level distance=1cm,
            level 1/.style={sibling distance=3cm},
            level 2/.style={sibling distance=1.5cm},
            level 3/.style={sibling distance=1cm}]
            \node {0}
                child {
                    node {1}
                    child {
                        node {3}
                        child {
                            node {7}
                        }
                        child {
                            node {8}
                        }
                    }
                    child {
                        node {4}
                        child {
                            node {9}
                        }
                    }
                }
                child {
                    node {2}
                    child {
                        node {5}
                    }
                    child {
                        node {6}
                    }
                };
    \end{tikzpicture}
    \begin{tabular}{ | c | c | c | c | c | c | c | c | c | c | }
        \hline
        0 & 1 & 2 & 3 & 4 & 5 & 6 & 7 & 8 & 9 \\ \hline
    \end{tabular}
    \caption{the heap is correctly sorted}
    \end{subfigure}
\end{figure}
\end{center}
%\begin{center}
    \begin{tikzpicture}[level distance=1cm,
        level 1/.style={sibling distance=3cm},
        level 2/.style={sibling distance=1.5cm},
        level 3/.style={sibling distance=1cm}]
        \node {7}
            child {
                node {1}
                child {
                    node {4}
                    child {
                        node {3}
                    }
                    child {
                        node {0}
                    }
                }
                child {
                    node {9}
                    child {
                        node {6}
                    }
                }
            }
            child {
                node {5}
                child {
                    node {2}
                }
                child {
                    node {8}
                }
            };
    \end{tikzpicture}
    \begin{tikzpicture}[level distance=1cm,
        level 1/.style={sibling distance=3cm},
        level 2/.style={sibling distance=1.5cm},
        level 3/.style={sibling distance=1cm}]
        \node {7}
            child {
                node {1}
                child {
                    node {4}
                    child {
                        node {3}
                    }
                    child {
                        node {0}
                    }
                }
                child {
                    node {9}
                    child {
                        node {6}
                    }
                }
            }
            child {
                node {\color{red}8}
                child {
                    node {2}
                }
                child {
                    node {\color{red}5}
                }
            };
    \end{tikzpicture}
    \begin{tikzpicture}[level distance=1cm,
        level 1/.style={sibling distance=3cm},
        level 2/.style={sibling distance=1.5cm},
        level 3/.style={sibling distance=1cm}]
        \node {7}
            child {
                node {\color{red}9}
                child {
                    node {4}
                    child {
                        node {3}
                    }
                    child {
                        node {0}
                    }
                }
                child {
                    node {\color{red}1}
                    child {
                        node {6}
                    }
                }
            }
            child {
                node {8}
                child {
                    node {2}
                }
                child {
                    node {5}
                }
            };
    \end{tikzpicture}
    \begin{tikzpicture}[level distance=1cm,
        level 1/.style={sibling distance=3cm},
        level 2/.style={sibling distance=1.5cm},
        level 3/.style={sibling distance=1cm}]
        \node {7}
            child {
                node {9}
                child {
                    node {4}
                    child {
                        node {3}
                    }
                    child {
                        node {0}
                    }
                }
                child {
                    node {\color{red}6}
                    child {
                        node {\color{red}1}
                    }
                }
            }
            child {
                node {8}
                child {
                    node {2}
                }
                child {
                    node {5}
                }
            };
    \end{tikzpicture}
    \begin{tikzpicture}[level distance=1cm,
        level 1/.style={sibling distance=3cm},
        level 2/.style={sibling distance=1.5cm},
        level 3/.style={sibling distance=1cm}]
        \node {\color{red}9}
            child {
                node {\color{red}7}
                child {
                    node {4}
                    child {
                        node {3}
                    }
                    child {
                        node {0}
                    }
                }
                child {
                    node {6}
                    child {
                        node {1}
                    }
                }
            }
            child {
                node {8}
                child {
                    node {2}
                }
                child {
                    node {5}
                }
            };
    \end{tikzpicture}
    \begin{tikzpicture}[level distance=1cm,
        level 1/.style={sibling distance=3cm},
        level 2/.style={sibling distance=1.5cm},
        level 3/.style={sibling distance=1cm}]
        \node {\color{red}1}
            child {
                node {7}
                child {
                    node {4}
                    child {
                        node {3}
                    }
                    child {
                        node {0}
                    }
                }
                child {
                    node {6}
                    child {
                        node {\color{DarkGreen}9}
                    }
                }
            }
            child {
                node {8}
                child {
                    node {2}
                }
                child {
                    node {5}
                }
            };
    \end{tikzpicture}
    \begin{tikzpicture}[level distance=1cm,
        level 1/.style={sibling distance=3cm},
        level 2/.style={sibling distance=1.5cm},
        level 3/.style={sibling distance=1cm}]
        \node {\color{red}8}
            child {
                node {7}
                child {
                    node {4}
                    child {
                        node {3}
                    }
                    child {
                        node {0}
                    }
                }
                child {
                    node {6}
                    child {
                        node {\color{DarkGreen}9}
                    }
                }
            }
            child {
                node {\color{red}1}
                child {
                    node {2}
                }
                child {
                    node {5}
                }
            };
    \end{tikzpicture}
    \begin{tikzpicture}[level distance=1cm,
        level 1/.style={sibling distance=3cm},
        level 2/.style={sibling distance=1.5cm},
        level 3/.style={sibling distance=1cm}]
        \node {8}
            child {
                node {7}
                child {
                    node {4}
                    child {
                        node {3}
                    }
                    child {
                        node {0}
                    }
                }
                child {
                    node {6}
                    child {
                        node {\color{DarkGreen}9}
                    }
                }
            }
            child {
                node {\color{red}5}
                child {
                    node {2}
                }
                child {
                    node {\color{red}1}
                }
            };
    \end{tikzpicture}
    \begin{tikzpicture}[level distance=1cm,
        level 1/.style={sibling distance=3cm},
        level 2/.style={sibling distance=1.5cm},
        level 3/.style={sibling distance=1cm}]
        \node {\color{red}0}
            child {
                node {7}
                child {
                    node {4}
                    child {
                        node {3}
                    }
                    child {
                        node {\color{DarkGreen}8}
                    }
                }
                child {
                    node {6}
                    child {
                        node {\color{DarkGreen}9}
                    }
                }
            }
            child {
                node {5}
                child {
                    node {2}
                }
                child {
                    node {1}
                }
            };
    \end{tikzpicture}
    \begin{tikzpicture}[level distance=1cm,
        level 1/.style={sibling distance=3cm},
        level 2/.style={sibling distance=1.5cm},
        level 3/.style={sibling distance=1cm}]
        \node {\color{red}7}
            child {
                node {\color{red}0}
                child {
                    node {4}
                    child {
                        node {3}
                    }
                    child {
                        node {\color{DarkGreen}8}
                    }
                }
                child {
                    node {6}
                    child {
                        node {\color{DarkGreen}9}
                    }
                }
            }
            child {
                node {5}
                child {
                    node {2}
                }
                child {
                    node {1}
                }
            };
    \end{tikzpicture}
    \begin{tikzpicture}[level distance=1cm,
        level 1/.style={sibling distance=3cm},
        level 2/.style={sibling distance=1.5cm},
        level 3/.style={sibling distance=1cm}]
        \node {7}
            child {
                node {\color{red}6}
                child {
                    node {4}
                    child {
                        node {3}
                    }
                    child {
                        node {\color{DarkGreen}8}
                    }
                }
                child {
                    node {\color{red}0}
                    child {
                        node {\color{DarkGreen}9}
                    }
                }
            }
            child {
                node {5}
                child {
                    node {2}
                }
                child {
                    node {1}
                }
            };
    \end{tikzpicture}
    \begin{tikzpicture}[level distance=1cm,
        level 1/.style={sibling distance=3cm},
        level 2/.style={sibling distance=1.5cm},
        level 3/.style={sibling distance=1cm}]
        \node {\color{red}3}
            child {
                node {6}
                child {
                    node {4}
                    child {
                        node {\color{DarkGreen}7}
                    }
                    child {
                        node {\color{DarkGreen}8}
                    }
                }
                child {
                    node {0}
                    child {
                        node {\color{DarkGreen}9}
                    }
                }
            }
            child {
                node {5}
                child {
                    node {2}
                }
                child {
                    node {1}
                }
            };
    \end{tikzpicture}
\end{center}
%\section{Worked Example 1}
\begin{center}
    \begin{tabular}{ | c | c | c | c | c | c | c | c | c | c || l | }
        \hline
        7 & 1 & 5 & 4 & 9 & 2 & 8 & 3 & 0 &\cellcolor{LightCyan}6 &(1) find the \color{cyan}pivot\\ \hline % i = -1
        \color{red}1 & \color{red}7 & 5 & 4 & 9 & 2 & 8 & 3 & 0 &\cellcolor{LightCyan}6 &(2) swap \(\color{red}7\) and \(\color{red}1\)\\ \hline % i = 0
        1 & \color{red}5 & \color{red}7 & 4 & 9 & 2 & 8 & 3 & 0 &\cellcolor{LightCyan}6 &(3) swap \(\color{red}7\) and \(\color{red}5\)\\ \hline % i = 1
        1 & 5 & \color{red}4 & \color{red}7 & 9 & 2 & 8 & 3 & 0 &\cellcolor{LightCyan}6 &(4) swap \(\color{red}7\) and \(\color{red}4\)\\ \hline % i = 2
        1 & 5 & 4 & \color{red}2 & 9 & \color{red} 7 & 8 & 3 & 0 &\cellcolor{LightCyan}6 &(5) swap \(\color{red}7\) and \(\color{red}2\)\\ \hline % i = 3
        1 & 5 & 4 & 2 & \color{red}3 & 7 & 8 & \color{red}9 & 0 &\cellcolor{LightCyan}6 &(6) swap \(\color{red}9\) and \(\color{red}3\)\\ \hline % i = 4
        1 & 5 & 4 & 2 & 3 & \color{red}0 & 8 & 9 & \color{red}7 &\cellcolor{LightCyan}6 &(7) swap \(\color{red}7\) and \(\color{red}0\)\\ \hline % i = 5
        1 & 5 & 4 & 2 & 3 & 0 & \color{cyan}6 & 9 & 7 & \color{red}8 &(8) swap \(\color{red}8\) and the {\color{cyan}pivot}\\ \hline % i = 6 p_i = 6
        1 & 5 & 4 & 2 & 3 & 0 & \cellcolor{LightGreen}6 & 9 & 7 & 8 &(9) \(\color{green}6\) is in the correct place \\ \hline \hline
        \multicolumn{11}{ | c | }{partition the sequence into \((1, 5, 4, 2, 3, 0)\) and \((9, 7, 8)\)} \\ \hline \hline
        \cellcolor{Amber}1 & \cellcolor{Amber}5 & \cellcolor{Amber}4 & \cellcolor{Amber}2 & \cellcolor{Amber}3 & \cellcolor{Amber}0 & \cellcolor{LightGreen}6 & \cellcolor{LightGrey}9 & \cellcolor{LightGrey}7 & \cellcolor{LightGrey}8 & sort {\color{DarkOrange}left side} \\ \hline
        1 & 5 & 4 & 2 & 3 & \cellcolor{LightCyan}0 & \cellcolor{LightGreen}6 & \cellcolor{LightGrey}9 & \cellcolor{LightGrey}7 & \cellcolor{LightGrey}8 &find the {\color{cyan}pivot} \\ \hline
        \color{cyan}0 & 5 & 4 & 2 & 3 & \color{red}1 & \cellcolor{LightGreen}6 & \cellcolor{LightGrey}9 & \cellcolor{LightGrey}7 & \cellcolor{LightGrey}8 &swap \(\color{red}1\) and the {\color{cyan}pivot}\\ \hline
        \cellcolor{LightGreen}0 & 5 & 4 & 2 & 3 & 1 & \cellcolor{LightGreen}6 & \cellcolor{LightGrey}9 & \cellcolor{LightGrey}7 & \cellcolor{LightGrey}8 &{\color{green}0} is in the correct place\\ \hline \hline
        \multicolumn{11}{ | c | }{partition the sequence into \(()\) and \((5, 4, 2, 3, 1)\)} \\ \hline \hline
        \cellcolor{LightGreen}0 & \cellcolor{LightGrey}5 & \cellcolor{LightGrey}4 & \cellcolor{LightGrey}2 & \cellcolor{LightGrey}3 & \cellcolor{LightGrey}1 & \cellcolor{LightGreen}6 & \cellcolor{LightGrey}9 & \cellcolor{LightGrey}7 & \cellcolor{LightGrey}8 &nothing to sort on the {\color{DarkOrange}left side}\\ \hline
        \cellcolor{LightGreen}0 & \cellcolor{Amber}5 & \cellcolor{Amber}4 & \cellcolor{Amber}2 & \cellcolor{Amber}3 & \cellcolor{Amber}1 & \cellcolor{LightGreen}6 & \cellcolor{LightGrey}9 & \cellcolor{LightGrey}7 & \cellcolor{LightGrey}8 &sort {\color{DarkOrange}right side}\\ \hline
        \cellcolor{LightGreen}0 & 5 & 4 & 2 & 3 & \cellcolor{LightCyan}1 & \cellcolor{LightGreen}6 & \cellcolor{LightGrey}9 & \cellcolor{LightGrey}7 & \cellcolor{LightGrey}8 &find the {\color{cyan}pivot} \\ \hline
        \cellcolor{LightGreen}0 & \color{cyan}1 & 4 & 2 & 3 & \color{red}5 & \cellcolor{LightGreen}6 & \cellcolor{LightGrey}9 & \cellcolor{LightGrey}7 & \cellcolor{LightGrey}8 &swap \(\color{red}5\) and the {\color{cyan}pivot} \\ \hline
        \cellcolor{LightGreen}0 & \cellcolor{LightGreen}1 & 4 & 2 & 3 & 5 & \cellcolor{LightGreen}6 & \cellcolor{LightGrey}9 & \cellcolor{LightGrey}7 & \cellcolor{LightGrey}8 &{\color{green}1} is in the correct place \\ \hline
        \cellcolor{LightGreen}0 & \cellcolor{LightGreen}1 & 4 & 2 & 3 & \cellcolor{LightCyan}5 & \cellcolor{LightGreen}6 & \cellcolor{LightGrey}9 & \cellcolor{LightGrey}7 & \cellcolor{LightGrey}8 &find the {\color{cyan}pivot} \\ \hline
        \cellcolor{LightGreen}0 & \cellcolor{LightGreen}1 & 4 & 2 & 3 & \cellcolor{LightGreen}5 & \cellcolor{LightGreen}6 & \cellcolor{LightGrey}9 & \cellcolor{LightGrey}7 & \cellcolor{LightGrey}8 &\(\color{green}5\) is in the correct place \\ \hline
        \cellcolor{LightGreen}0 & \cellcolor{LightGreen}1 & 4 & 2 & \cellcolor{LightCyan}3 & \cellcolor{LightGreen}5 & \cellcolor{LightGreen}6 & \cellcolor{LightGrey}9 & \cellcolor{LightGrey}7 & \cellcolor{LightGrey}8 &find the {\color{cyan}pivot} \\ \hline
        \cellcolor{LightGreen}0 & \cellcolor{LightGreen}1 & \color{red}2 & \color{red}4 & \cellcolor{LightCyan}3 & \cellcolor{LightGreen}5 & \cellcolor{LightGreen}6 & \cellcolor{LightGrey}9 & \cellcolor{LightGrey}7 & \cellcolor{LightGrey}8 &swap \(\color{red}4\) and \(\color{red}2\) \\ \hline
        \cellcolor{LightGreen}0 & \cellcolor{LightGreen}1 & \color{red}2 & \color{cyan}3 & \color{red}4 & \cellcolor{LightGreen}5 & \cellcolor{LightGreen}6 & \cellcolor{LightGrey}9 & \cellcolor{LightGrey}7 & \cellcolor{LightGrey}8 &swap \(\color{red}4\) and the {\color{cyan}pivot} \\ \hline
        \cellcolor{LightGreen}0 & \cellcolor{LightGreen}1 & 2 & \cellcolor{LightGreen}3 & 4 & \cellcolor{LightGreen}5 & \cellcolor{LightGreen}6 & \cellcolor{LightGrey}9 & \cellcolor{LightGrey}7 & \cellcolor{LightGrey}8 &\(\color{LightGreen}3\) is in the correct place \\ \hline
    \end{tabular}
\end{center}

%\begin{center}
    \begin{tabular}{ | c | c | c | c | c | c | c | c | c | c | l | }
      \hline
      7 & 1 & 5 & 4 & 9 & 2 & 8 & 3 & 0 &\cellcolor{LightCyan}6 &find the \color{cyan}pivot\\ \hline % i = -1
      \color{red}1 & \color{red}7 & 5 & 4 & 9 & 2 & 8 & 3 & 0 &\cellcolor{LightCyan}6 &swap \(\color{red}7\) and \(\color{red}1\)\\ \hline % i = 0
      1 & \color{red}5 & \color{red}7 & 4 & 9 & 2 & 8 & 3 & 0 &\cellcolor{LightCyan}6 &swap \(\color{red}7\) and \(\color{red}5\)\\ \hline % i = 1
      1 & 5 & \color{red}4 & \color{red}7 & 9 & 2 & 8 & 3 & 0 &\cellcolor{LightCyan}6 &swap \(\color{red}7\) and \(\color{red}4\)\\ \hline % i = 2
      1 & 5 & 4 & \color{red}2 & 9 & \color{red} 7 & 8 & 3 & 0 &\cellcolor{LightCyan}6 &swap \(\color{red}7\) and \(\color{red}2\)\\ \hline % i = 3
      1 & 5 & 4 & 2 & \color{red}3 & 7 & 8 & \color{red}9 & 0 &\cellcolor{LightCyan}6 &swap \(\color{red}9\) and \(\color{red}3\)\\ \hline % i = 4
      1 & 5 & 4 & 2 & 3 & \color{red}0 & 8 & 9 & \color{red}7 &\cellcolor{LightCyan}6 &swap \(\color{red}7\) and \(\color{red}0\)\\ \hline % i = 5
      1 & 5 & 4 & 2 & 3 & 0 & \color{cyan}6 & 9 & 7 & \color{red}8 &swap \(\color{red}8\) and the {\color{cyan}pivot}\\ \hline % i = 6 p_i = 6
    \end{tabular}
\end{center}

Now, the pivot \(6\) is on the right place and every element on the left side is smaller and every element on the right side is larger than the pivot.

\begin{center}
    \begin{tabular}{ | c | c | c | c | c | c || c || c | c | c | }
        \hline
        1 & 5 & 4 & 2 & 3 & 0 & \cellcolor{LightCyan}6 & 9 & 7 & 8 \\ \hline
    \end{tabular}
\end{center}

We partition the sequence into two smaller ones and apply the algorithm on each.

\begin{center}
    \begin{tabular}{ | c | c | c | c | c | c | l | }
        \hline
        1 & 5 & 4 & 2 & 3 & \cellcolor{LightCyan}0 & find the {\color{cyan}pivot}\\ \hline
        \color{cyan}0 & 5 & 4 & 2 & 3 & \color{red}1 & swap \(\color{red}1\) and the {\color{cyan}pivot}\\ \hline
    \end{tabular}
\end{center}

The pivot \(0\) is correctly placed.

\begin{center}
    \begin{tabular}{ || c || c | c | c | c | c | }
        \hline
        \cellcolor{LightCyan}0 & 5 & 4 & 2 & 3 & 1 \\ \hline
    \end{tabular}
\end{center}

Since there is no left side of the pivot, we proceed with the right side.

\begin{center}
    \begin{tabular}{ | c | c | c | c | c | l | }
        \hline
        5 & 4 & 2 & 3 & \cellcolor{LightCyan}1 & find the {\color{cyan}pivot} \\ \hline
        \color{cyan}1 & 4 & 2 & 3 & \color{red}5 & swap \(\color{red}5\) and the {\color{cyan}pivot} \\ \hline
    \end{tabular}
\end{center}

Again, \(1\) is placed correctly in the far left. The following sequence is left.

\begin{center}
    \begin{tabular}{ | c | c | c | c | c | }
        \hline
        \cellcolor{LightCyan}1 & 4 & 2 & 3 & 5 \\ \hline
    \end{tabular}
\end{center}

Now we have

\begin{center}
    \begin{tabular}{ | c | c | c | c | l | }
        \hline
        4 & 2 & 3 & \cellcolor{LightCyan}5 & find the {\color{cyan}pivot} \\ \hline
    \end{tabular}
\end{center}

since the pivot \(5\) is already correctly placed, there is no swapping to do. We continue with

\begin{center}
    \begin{tabular}{ | c | c | c | l | }
        \hline
        4 & 2 & \cellcolor{LightCyan}3 & find the {\color{cyan}pivot} \\ \hline
        \color{red}2 & \color{red}4 & \cellcolor{LightCyan}3 & swap \(\color{red}4\) and \(\color{red}2\) \\ \hline
        2 & \color{cyan}3 & \color{red}4 & swap \(\color{red}4\) and the {\color{cyan}pivot} \\ \hline
    \end{tabular}
\end{center}

After this, the left side of the inital partition is correctly sorted.

\begin{center}
    \begin{tabular}{ | c | c | c | c | c | c || c || c | c | c | }
        \hline
        0 & 1 & 2 & 3 & 4 & 5 & \cellcolor{LightCyan}6 & 9 & 7 & 8 \\ \hline
    \end{tabular}
\end{center}

We continue with the right side.

\begin{center}
    \begin{tabular}{ | c | c | c | l |}
        \hline
        9 & 7 & \cellcolor{LightCyan}8 & find the {\color{cyan}pivot} \\ \hline
        \color{red}7 & \color{red}9 & \cellcolor{LightCyan}8 & swap \(\color{red}9\) and \(\color{red}7\) \\ \hline
        7 & \color{cyan}8 & \color{red}9 & swap \(\color{red}9\) and the {\color{cyan}pivot} \\ \hline
    \end{tabular}
\end{center}

At the end of the algorithm we have the correctly sorted list.

\begin{center}
    \begin{tabular}{ | c | c | c | c | c | c | c | c | c | c | }
        \hline
        0 & 1 & 2 & 3 & 4 & 5 & 6 & 7 & 8 & 9 \\ \hline
    \end{tabular}
\end{center}
\newpage
\section{sorting\_analysis.py User Manual}
The companion Python module to this paper \texttt{sort\_analysis.py} was written to verify experimentally the aforementioned theoretical results.

\includegraphics{images/help_command.png}
\includegraphics{images/quicksort_demo.png}
\includegraphics{images/heapsort_demo.png}
\section{sorting\_analysis.py API}
\subsection{sort\_A(\_list)}
These following functions can be imported for custom use.
\subsubsection*{Arguments}
\begin{enumerate}
    \item \texttt{\_list} (list): the list to be sorted, each element must implement the comparison operator for < (\_\_lt\_\_)
\end{enumerate}
\subsubsection*{Returns}
\begin{itemize}
    \item (int, int): 2-tuple with first value the number of operations (compares + swaps) and second value the time consumed in milliseconds.
\end{itemize}
\subsubsection*{Description}
Performs quicksort on given parameter \_list.
\subsection{sort\_B(\_list)}
\subsubsection*{Arguments}
\begin{enumerate}
    \item \texttt{\_list} (list): the list to be sorted, each element must implement the comparison operator for < (\_\_lt\_\_)
\end{enumerate}
\subsubsection*{Returns}
\begin{itemize}
    \item (int, int): 2-tuple with first value the number of operations (compares + swaps) and second value the time consumed in milliseconds.
\end{itemize}
\subsubsection*{Description}
Performs heapsort on given parameter \_list.
\subsection{read\_words\_from\_file(\_filename)}
\subsubsection*{Arguments}
\begin{enumerate}
    \item \texttt{\_filename} (str): Path to local file.
\end{enumerate}
\subsubsection*{Returns}
\begin{itemize}
    \item (list): List of words (as strings).
\end{itemize}
\subsubsection*{Description}
Opens given filename, reads its full contens, and splits it into words, delimited by whitesapce.
\section{Experiments}\label{sec:exp}
We will test quicksort and heapsort by sorting 5 lists all containing exactly 100 integers from 1 to 100. 
\begin{center}
\begin{figure}[h]

    \begin{tabular}{ | l | l | l | l | l | l |}
        \hline
        & time elapsed (in s)\footnote{Time elapsed on the computer used for the experiment. This can vary depending on many different factors and should not be weighted too much importance.} & compares & swaps & iteration & recursion \\ \hline
        \textbf{Test 1} & & & & & \\ \hline
        quicksort & 171.30 & 1474 & 1288 & 1474 & 34 \\ \hline
        heapsort & 18.49 & 1031 & 600 & 150 & 7 \\ \hline
        \textbf{Test 2} & & & & & \\ \hline
        quicksort & 24.00 & 1022 & 515 & 1022 & 20 \\ \hline
        heapsort & 23.55 & 962 & 516 & 151 & 7 \\ \hline
        \textbf{Test 3} & & & & & \\ \hline
        quicksort & 5.75 & 672 & 381 & 647 & 12 \\ \hline
        heapsort & 18.19 & 1033 & 581 & 150 & 7 \\ \hline
        \textbf{Test 4} & & & & & \\ \hline
        quicksort & 3.16 & 576 & 343 & 576 & 11 \\ \hline
        heapsort & 18.79 & 1039 & 588 & 150 & 7 \\ \hline
        \textbf{Test 5} & & & & & \\ \hline
        quicksort & 4.40 & 749 & 414 & 749 & 16 \\ \hline
        heapsort & 16.23 & 1004 & 559 & 150 & 7 \\ \hline
    \end{tabular}
    \caption{The results of the python module. For the input lists of the tests, see the appendix.}\label{test}
\end{figure}
\end{center}
Test 1 was a list of integers already sorted (1, 2, 3, ...) while the list used in test 2 was reversed sorted (i.e. 100, 99, 98, ...). The other three lists were a list of random integers from 1 to 100 each occuring once. See figure \ref{test} for detailed results.
\section{Conclusion and the Future}\label{future}
One of the immediate optimization idea for quicksort is to make the choice of the pivot more intelligently. We have seen that with the way our current set up for quicksort lists which are already sorted are one of the worst cases. In practice, having the worst case for already sorted lists is not desireble. Median of three, for example, could be a way of choosing the pivot more smartly which could be implemented in the future.

As we have already mentioned before, the strenth of heapsort lies in its consistent performance while the weakness of quicksort is its worst case. Therefore, it would make sense to combine these two algorithms to cover up each of their weaknesses. Indeed, such an algorithm is called introsort which would be the next goal to achive after implementing quicksort and heapsort.
\newpage
\begin{thebibliography}{9}
        \bibitem{bib:introductiontoalgorithms} 
        Thomas H. Cormen, Charles E. Leiserson, Ronald L. Rivest, and Clifford Stein. 
        \textit{Introduction to Algorithms}. 
        The MIT Press, Second Edition, Cambridge, Massachusetts, 2003.
        \bibitem{bib:thealgorithmdesignmanual}
        Steven S. Skiena.
        \textit{The Algorithm Design Manual}
        Springer, Second Edition, 2008.
%\begin{center}
    \begin{tikzpicture}[level distance=1cm,
        level 1/.style={sibling distance=3cm},
        level 2/.style={sibling distance=1.5cm},
        level 3/.style={sibling distance=1cm}]
        \node {7}
            child {
                node {1}
                child {
                    node {4}
                    child {
                        node {3}
                    }
                    child {
                        node {0}
                    }
                }
                child {
                    node {9}
                    child {
                        node {6}
                    }
                }
            }
            child {
                node {5}
                child {
                    node {2}
                }
                child {
                    node {8}
                }
            };
    \end{tikzpicture}
    \begin{tikzpicture}[level distance=1cm,
        level 1/.style={sibling distance=3cm},
        level 2/.style={sibling distance=1.5cm},
        level 3/.style={sibling distance=1cm}]
        \node {7}
            child {
                node {1}
                child {
                    node {4}
                    child {
                        node {3}
                    }
                    child {
                        node {0}
                    }
                }
                child {
                    node {9}
                    child {
                        node {6}
                    }
                }
            }
            child {
                node {\color{red}8}
                child {
                    node {2}
                }
                child {
                    node {\color{red}5}
                }
            };
    \end{tikzpicture}
    \begin{tikzpicture}[level distance=1cm,
        level 1/.style={sibling distance=3cm},
        level 2/.style={sibling distance=1.5cm},
        level 3/.style={sibling distance=1cm}]
        \node {7}
            child {
                node {\color{red}9}
                child {
                    node {4}
                    child {
                        node {3}
                    }
                    child {
                        node {0}
                    }
                }
                child {
                    node {\color{red}1}
                    child {
                        node {6}
                    }
                }
            }
            child {
                node {8}
                child {
                    node {2}
                }
                child {
                    node {5}
                }
            };
    \end{tikzpicture}
    \begin{tikzpicture}[level distance=1cm,
        level 1/.style={sibling distance=3cm},
        level 2/.style={sibling distance=1.5cm},
        level 3/.style={sibling distance=1cm}]
        \node {7}
            child {
                node {9}
                child {
                    node {4}
                    child {
                        node {3}
                    }
                    child {
                        node {0}
                    }
                }
                child {
                    node {\color{red}6}
                    child {
                        node {\color{red}1}
                    }
                }
            }
            child {
                node {8}
                child {
                    node {2}
                }
                child {
                    node {5}
                }
            };
    \end{tikzpicture}
    \begin{tikzpicture}[level distance=1cm,
        level 1/.style={sibling distance=3cm},
        level 2/.style={sibling distance=1.5cm},
        level 3/.style={sibling distance=1cm}]
        \node {\color{red}9}
            child {
                node {\color{red}7}
                child {
                    node {4}
                    child {
                        node {3}
                    }
                    child {
                        node {0}
                    }
                }
                child {
                    node {6}
                    child {
                        node {1}
                    }
                }
            }
            child {
                node {8}
                child {
                    node {2}
                }
                child {
                    node {5}
                }
            };
    \end{tikzpicture}
    \begin{tikzpicture}[level distance=1cm,
        level 1/.style={sibling distance=3cm},
        level 2/.style={sibling distance=1.5cm},
        level 3/.style={sibling distance=1cm}]
        \node {\color{red}1}
            child {
                node {7}
                child {
                    node {4}
                    child {
                        node {3}
                    }
                    child {
                        node {0}
                    }
                }
                child {
                    node {6}
                    child {
                        node {\color{DarkGreen}9}
                    }
                }
            }
            child {
                node {8}
                child {
                    node {2}
                }
                child {
                    node {5}
                }
            };
    \end{tikzpicture}
    \begin{tikzpicture}[level distance=1cm,
        level 1/.style={sibling distance=3cm},
        level 2/.style={sibling distance=1.5cm},
        level 3/.style={sibling distance=1cm}]
        \node {\color{red}8}
            child {
                node {7}
                child {
                    node {4}
                    child {
                        node {3}
                    }
                    child {
                        node {0}
                    }
                }
                child {
                    node {6}
                    child {
                        node {\color{DarkGreen}9}
                    }
                }
            }
            child {
                node {\color{red}1}
                child {
                    node {2}
                }
                child {
                    node {5}
                }
            };
    \end{tikzpicture}
    \begin{tikzpicture}[level distance=1cm,
        level 1/.style={sibling distance=3cm},
        level 2/.style={sibling distance=1.5cm},
        level 3/.style={sibling distance=1cm}]
        \node {8}
            child {
                node {7}
                child {
                    node {4}
                    child {
                        node {3}
                    }
                    child {
                        node {0}
                    }
                }
                child {
                    node {6}
                    child {
                        node {\color{DarkGreen}9}
                    }
                }
            }
            child {
                node {\color{red}5}
                child {
                    node {2}
                }
                child {
                    node {\color{red}1}
                }
            };
    \end{tikzpicture}
    \begin{tikzpicture}[level distance=1cm,
        level 1/.style={sibling distance=3cm},
        level 2/.style={sibling distance=1.5cm},
        level 3/.style={sibling distance=1cm}]
        \node {\color{red}0}
            child {
                node {7}
                child {
                    node {4}
                    child {
                        node {3}
                    }
                    child {
                        node {\color{DarkGreen}8}
                    }
                }
                child {
                    node {6}
                    child {
                        node {\color{DarkGreen}9}
                    }
                }
            }
            child {
                node {5}
                child {
                    node {2}
                }
                child {
                    node {1}
                }
            };
    \end{tikzpicture}
    \begin{tikzpicture}[level distance=1cm,
        level 1/.style={sibling distance=3cm},
        level 2/.style={sibling distance=1.5cm},
        level 3/.style={sibling distance=1cm}]
        \node {\color{red}7}
            child {
                node {\color{red}0}
                child {
                    node {4}
                    child {
                        node {3}
                    }
                    child {
                        node {\color{DarkGreen}8}
                    }
                }
                child {
                    node {6}
                    child {
                        node {\color{DarkGreen}9}
                    }
                }
            }
            child {
                node {5}
                child {
                    node {2}
                }
                child {
                    node {1}
                }
            };
    \end{tikzpicture}
    \begin{tikzpicture}[level distance=1cm,
        level 1/.style={sibling distance=3cm},
        level 2/.style={sibling distance=1.5cm},
        level 3/.style={sibling distance=1cm}]
        \node {7}
            child {
                node {\color{red}6}
                child {
                    node {4}
                    child {
                        node {3}
                    }
                    child {
                        node {\color{DarkGreen}8}
                    }
                }
                child {
                    node {\color{red}0}
                    child {
                        node {\color{DarkGreen}9}
                    }
                }
            }
            child {
                node {5}
                child {
                    node {2}
                }
                child {
                    node {1}
                }
            };
    \end{tikzpicture}
    \begin{tikzpicture}[level distance=1cm,
        level 1/.style={sibling distance=3cm},
        level 2/.style={sibling distance=1.5cm},
        level 3/.style={sibling distance=1cm}]
        \node {\color{red}3}
            child {
                node {6}
                child {
                    node {4}
                    child {
                        node {\color{DarkGreen}7}
                    }
                    child {
                        node {\color{DarkGreen}8}
                    }
                }
                child {
                    node {0}
                    child {
                        node {\color{DarkGreen}9}
                    }
                }
            }
            child {
                node {5}
                child {
                    node {2}
                }
                child {
                    node {1}
                }
            };
    \end{tikzpicture}
\end{center}
\end{thebibliography}

\section{Appendix}
The following five lits of integers were used in section \ref{sec:exp}, experiments.

\noindent\textbf{Test 1 (sorted)}

\noindent 1 2 3 4 5 6 7 8 9 10 11 12 13 14 15 16 17 18 19 20 21 22 23 24 25 26 27 28 29 30 31 32 33 34 35 36 37 38 39 40 41 42 43 44 45 46 47 48 49 50 51 52 53 54 55 56 57 58 59 60 61 62 63 64 65 66 67 68 69 70 71 72 73 74 75 76 77 78 79 80 81 82 83 84 85 86 87 88 89 90 91 92 93 94 95 96 97 98 99 100

\noindent\textbf{Test 2 (reversed sorted)}

\noindent 100 99 98 97 96 95 94 93 92 91 90 89 88 87 86 85 84 83 82 81 80 79 78 77 76 75 74 73 72 71 70 69 68 67 66 65 64 63 62 61 60 59 58 57 56 55 54 53 52 51 50 49 48 47 46 45 44 43 42 41 40 39 38 37 36 35 34 33 32 31 30 29 28 27 26 25 24 23 22 21 20 19 18 17 16 15 14 13 12 11 10 9 8 7 6 5 4 3 2 1 0



\noindent\textbf{Test 3 (random)}

\noindent 60 26 76 82 95 59 27 20 42 67 11 57 39 1 45 61 3 65 53 15 49 77 43 99 97 80 48 79 17 93 73 89 64 88 25 75 7 63 21 58 10 9 32 18 29 69 90 96 2 37 4 19 66 16 81 35 30 51 14 33 70 5 94 56 41 44 38 54 23 91 6 84 86 68 13 62 31 71 87 72 12 34 47 50 98 55 78 83 46 85 74 40 22 28 8 36 52 100 92 24



\noindent\textbf{Test 4 (random)}

\noindent 66 12 80 61 19 47 72 10 18 82 21 59 43 76 31 34 81 87 11 84 71 78 5 29 60 24 89 41 13 68 74 79 48 38 25 73 56 62 86 26 33 50 93 69 22 70 27 98 95 64 6 23 97 90 7 96 9 20 8 17 46 15 35 85 94 4 52 100 44 92 3 30 54 49 32 53 88 83 63 91 36 37 77 57 16 55 51 1 14 75 67 2 99 39 28 40 65 45 58 42

\noindent\textbf{Test 5 (random)}

\noindent 93 21 42 55 94 89 82 16 6 67 41 97 30 88 22 75 9 17 3 49 62 80 40 48 63 33 92 86 66 50 84 45 18 31 32 85 24 37 44 83 43 95 51 68 29 81 98 26 100 27 58 76 72 11 23 59 19 46 28 60 53 8 74 69 10 7 1 56 20 54 36 77 39 47 71 87 38 4 78 34 2 96 61 35 57 70 64 15 14 73 5 52 65 79 25 99 12 91 90 13
\end{document}