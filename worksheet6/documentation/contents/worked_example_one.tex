\section{Worked Example 1}
Color Key
\begin{itemize}
    \item partitions to be sorted in the following steps are marked with {\color{Amber}orange}
    \item partitions currently ignored are marked with {\color{gray}gray}
    \item pivots are marked with {\color{cyan}teal}
    \item elements which were swapped are in {\color{red}red}
\end{itemize}

\begin{center}
    \begin{longtable}{ | c | c | c | c | c | c | c | c | c | c || l | }
        \hline
        7 & 1 & 5 & 4 & 9 & 2 & 8 & 3 & 0 & 6 &(0) the initial state \\ \hline
        7 & 1 & 5 & 4 & 9 & 2 & 8 & 3 & 0 &\cellcolor{LightCyan}6 &(1) choose \color{cyan}pivot\\ \hline % i = -1
        \color{red}1 & \color{red}7 & 5 & 4 & 9 & 2 & 8 & 3 & 0 &\cellcolor{LightCyan}6 &(2) swap \(\color{red}7\) and \(\color{red}1\)\\ \hline % i = 0
        1 & \color{red}5 & \color{red}7 & 4 & 9 & 2 & 8 & 3 & 0 &\cellcolor{LightCyan}6 &(3) swap \(\color{red}7\) and \(\color{red}5\)\\ \hline % i = 1
        1 & 5 & \color{red}4 & \color{red}7 & 9 & 2 & 8 & 3 & 0 &\cellcolor{LightCyan}6 &(4) swap \(\color{red}7\) and \(\color{red}4\)\\ \hline % i = 2
        1 & 5 & 4 & \color{red}2 & 9 & \color{red} 7 & 8 & 3 & 0 &\cellcolor{LightCyan}6 &(5) swap \(\color{red}7\) and \(\color{red}2\)\\ \hline % i = 3
        1 & 5 & 4 & 2 & \color{red}3 & 7 & 8 & \color{red}9 & 0 &\cellcolor{LightCyan}6 &(6) swap \(\color{red}9\) and \(\color{red}3\)\\ \hline % i = 4
        1 & 5 & 4 & 2 & 3 & \color{red}0 & 8 & 9 & \color{red}7 &\cellcolor{LightCyan}6 &(7) swap \(\color{red}7\) and \(\color{red}0\)\\ \hline % i = 5
        1 & 5 & 4 & 2 & 3 & 0 & \color{cyan}6 & 9 & 7 & \color{red}8 &(8) swap \(\color{red}8\) and the {\color{cyan}pivot}\\ \hline % i = 6 p_i = 6
        1 & 5 & 4 & 2 & 3 & 0 & \cellcolor{LightGreen}6 & 9 & 7 & 8 &(9) \(\color{green}6\) is in the correct place \\ \hhline{===========}
        \multicolumn{11}{ | c | }{partition the sequence into \((1, 5, 4, 2, 3, 0)\) and \((9, 7, 8)\)} \\ \hhline{===========}
        \cellcolor{Amber}1 & \cellcolor{Amber}5 & \cellcolor{Amber}4 & \cellcolor{Amber}2 & \cellcolor{Amber}3 & \cellcolor{Amber}0 & \cellcolor{LightGreen}6 & \cellcolor{LightGrey}9 & \cellcolor{LightGrey}7 & \cellcolor{LightGrey}8 &(10) sort {\color{DarkOrange}left side} \\ \hline
        1 & 5 & 4 & 2 & 3 & \cellcolor{LightCyan}0 & \cellcolor{LightGreen}6 & \cellcolor{LightGrey}9 & \cellcolor{LightGrey}7 & \cellcolor{LightGrey}8 &(11) choose {\color{cyan}pivot} \\ \hline
        \color{cyan}0 & 5 & 4 & 2 & 3 & \color{red}1 & \cellcolor{LightGreen}6 & \cellcolor{LightGrey}9 & \cellcolor{LightGrey}7 & \cellcolor{LightGrey}8 &(12) swap \(\color{red}1\) and the {\color{cyan}pivot}\\ \hline
        \cellcolor{LightGreen}0 & 5 & 4 & 2 & 3 & 1 & \cellcolor{LightGreen}6 & \cellcolor{LightGrey}9 & \cellcolor{LightGrey}7 & \cellcolor{LightGrey}8 &(13) {\color{green}0} is in the correct place\\ \hhline{===========}
        \multicolumn{11}{ | c | }{partition the sequence into \(()\) and \((5, 4, 2, 3, 1)\)} \\ \hhline{===========}
        \cellcolor{LightGreen}0 & \cellcolor{LightGrey}5 & \cellcolor{LightGrey}4 & \cellcolor{LightGrey}2 & \cellcolor{LightGrey}3 & \cellcolor{LightGrey}1 & \cellcolor{LightGreen}6 & \cellcolor{LightGrey}9 & \cellcolor{LightGrey}7 & \cellcolor{LightGrey}8 &(14) nothing to sort on the {\color{DarkOrange}left side}\\ \hline
        \cellcolor{LightGreen}0 & \cellcolor{Amber}5 & \cellcolor{Amber}4 & \cellcolor{Amber}2 & \cellcolor{Amber}3 & \cellcolor{Amber}1 & \cellcolor{LightGreen}6 & \cellcolor{LightGrey}9 & \cellcolor{LightGrey}7 & \cellcolor{LightGrey}8 &(15) sort {\color{DarkOrange}right side}\\ \hline
        \cellcolor{LightGreen}0 & 5 & 4 & 2 & 3 & \cellcolor{LightCyan}1 & \cellcolor{LightGreen}6 & \cellcolor{LightGrey}9 & \cellcolor{LightGrey}7 & \cellcolor{LightGrey}8 &(16) choose {\color{cyan}pivot} \\ \hline
        \cellcolor{LightGreen}0 & \color{cyan}1 & 4 & 2 & 3 & \color{red}5 & \cellcolor{LightGreen}6 & \cellcolor{LightGrey}9 & \cellcolor{LightGrey}7 & \cellcolor{LightGrey}8 &(17) swap \(\color{red}5\) and the {\color{cyan}pivot} \\ \hline
        \cellcolor{LightGreen}0 & \cellcolor{LightGreen}1 & 4 & 2 & 3 & 5 & \cellcolor{LightGreen}6 & \cellcolor{LightGrey}9 & \cellcolor{LightGrey}7 & \cellcolor{LightGrey}8 &(18) {\color{green}1} is in the correct place \\ \hhline{===========}
        \multicolumn{11}{ | c | }{partition the sequence into \(()\) and \((4, 2, 3, 5)\)} \\ \hhline{===========}
        \cellcolor{LightGreen}0 & \cellcolor{LightGreen}1 & \cellcolor{LightGrey}4 & \cellcolor{LightGrey}2 & \cellcolor{LightGrey}3 & \cellcolor{LightGrey}5 & \cellcolor{LightGreen}6 & \cellcolor{LightGrey}9 & \cellcolor{LightGrey}7 & \cellcolor{LightGrey}8 &(19) nothing to sort on the {\color{DarkOrange}left side}\\ \hline
        \cellcolor{LightGreen}0 & \cellcolor{LightGreen}1 & \cellcolor{Amber}4 & \cellcolor{Amber}2 & \cellcolor{Amber}3 & \cellcolor{Amber}5 & \cellcolor{LightGreen}6 & \cellcolor{LightGrey}9 & \cellcolor{LightGrey}7 & \cellcolor{LightGrey}8 &(20) sort {\color{DarkOrange}right side}\\ \hline
        \cellcolor{LightGreen}0 & \cellcolor{LightGreen}1 & 4 & 2 & 3 & \cellcolor{LightCyan}5 & \cellcolor{LightGreen}6 & \cellcolor{LightGrey}9 & \cellcolor{LightGrey}7 & \cellcolor{LightGrey}8 &(21) choose {\color{cyan}pivot} \\ \hline
        \cellcolor{LightGreen}0 & \cellcolor{LightGreen}1 & 4 & 2 & 3 & \cellcolor{LightGreen}5 & \cellcolor{LightGreen}6 & \cellcolor{LightGrey}9 & \cellcolor{LightGrey}7 & \cellcolor{LightGrey}8 &(22) \(\color{green}5\) is in the correct place \\ \hhline{===========}
        \multicolumn{11}{ | c | }{partition the sequence into \((4, 2, 3)\) and \(()\)} \\ \hhline{===========}
        \cellcolor{LightGreen}0 & \cellcolor{LightGreen}1 & \cellcolor{Amber}4 & \cellcolor{Amber}2 & \cellcolor{Amber}3 & \cellcolor{LightGreen}5 & \cellcolor{LightGreen}6 & \cellcolor{LightGrey}9 & \cellcolor{LightGrey}7 & \cellcolor{LightGrey}8 &(23) sort {\color{DarkOrange}left side} \\ \hline
        \cellcolor{LightGreen}0 & \cellcolor{LightGreen}1 & 4 & 2 & \cellcolor{LightCyan}3 & \cellcolor{LightGreen}5 & \cellcolor{LightGreen}6 & \cellcolor{LightGrey}9 & \cellcolor{LightGrey}7 & \cellcolor{LightGrey}8 &(24) choose {\color{cyan}pivot} \\ \hline
        \cellcolor{LightGreen}0 & \cellcolor{LightGreen}1 & \color{red}2 & \color{red}4 & \cellcolor{LightCyan}3 & \cellcolor{LightGreen}5 & \cellcolor{LightGreen}6 & \cellcolor{LightGrey}9 & \cellcolor{LightGrey}7 & \cellcolor{LightGrey}8 &(25) swap \(\color{red}4\) and \(\color{red}2\) \\ \hline
        \cellcolor{LightGreen}0 & \cellcolor{LightGreen}1 & 2 & \color{cyan}3 & \color{red}4 & \cellcolor{LightGreen}5 & \cellcolor{LightGreen}6 & \cellcolor{LightGrey}9 & \cellcolor{LightGrey}7 & \cellcolor{LightGrey}8 &(26) swap \(\color{red}4\) and the {\color{cyan}pivot} \\ \hline
        \cellcolor{LightGreen}0 & \cellcolor{LightGreen}1 & 2 & \cellcolor{LightGreen}3 & 4 & \cellcolor{LightGreen}5 & \cellcolor{LightGreen}6 & \cellcolor{LightGrey}9 & \cellcolor{LightGrey}7 & \cellcolor{LightGrey}8 &(27) \(\color{LightGreen}3\) is in the correct place \\ \hline
        \cellcolor{LightGreen}0 & \cellcolor{LightGreen}1 & \cellcolor{LightGrey}2 & \cellcolor{LightGreen}3 & \cellcolor{LightGrey}4 & \cellcolor{LightGreen}5 & \cellcolor{LightGreen}6 & \cellcolor{LightGrey}9 & \cellcolor{LightGrey}7 & \cellcolor{LightGrey}8 &(28) nothing to sort on the {\color{DarkOrange}right side} \\ \hhline{===========}
        \multicolumn{11}{ | c | }{partition the sequence into \((2)\) and \((4)\)} \\ \hhline{===========}
        \cellcolor{LightGreen}0 & \cellcolor{LightGreen}1 & \cellcolor{Amber}2 & \cellcolor{LightGreen}3 & \cellcolor{LightGrey}4 & \cellcolor{LightGreen}5 & \cellcolor{LightGreen}6 & \cellcolor{LightGrey}9 & \cellcolor{LightGrey}7 & \cellcolor{LightGrey}8 &(27) sort {\color{DarkOrange}left side} \\ \hline
        \cellcolor{LightGreen}0 & \cellcolor{LightGreen}1 & \cellcolor{LightGreen}2 & \cellcolor{LightGreen}3 & \cellcolor{LightGrey}4 & \cellcolor{LightGreen}5 & \cellcolor{LightGreen}6 & \cellcolor{LightGrey}9 & \cellcolor{LightGrey}7 & \cellcolor{LightGrey}8 &(28) \(\color{green}2\) is in the correct place \\ \hline
        \cellcolor{LightGreen}0 & \cellcolor{LightGreen}1 & \cellcolor{LightGreen}2 & \cellcolor{LightGreen}3 & \cellcolor{Amber}4 & \cellcolor{LightGreen}5 & \cellcolor{LightGreen}6 & \cellcolor{LightGrey}9 & \cellcolor{LightGrey}7 & \cellcolor{LightGrey}8 &(29) sort {\color{DarkOrange}right side} \\ \hline
        \cellcolor{LightGreen}0 & \cellcolor{LightGreen}1 & \cellcolor{LightGreen}2 & \cellcolor{LightGreen}3 & \cellcolor{LightGreen}4 & \cellcolor{LightGreen}5 & \cellcolor{LightGreen}6 & \cellcolor{LightGrey}9 & \cellcolor{LightGrey}7 & \cellcolor{LightGrey}8 &(30) \(\color{green}4\) is in the correct place \\ \hhline{===========} \hhline{===========}
        \cellcolor{LightGreen}0 & \cellcolor{LightGreen}1 & \cellcolor{LightGreen}2 & \cellcolor{LightGreen}3 & \cellcolor{LightGreen}4 & \cellcolor{LightGreen}5 & \cellcolor{LightGreen}6 & \cellcolor{Amber}9 & \cellcolor{Amber}7 & \cellcolor{Amber}8 &(31) sort {\color{DarkOrange}right side} \\ \hline
        \cellcolor{LightGreen}0 & \cellcolor{LightGreen}1 & \cellcolor{LightGreen}2 & \cellcolor{LightGreen}3 & \cellcolor{LightGreen}4 & \cellcolor{LightGreen}5 & \cellcolor{LightGreen}6 & 9 & 7 & \cellcolor{LightCyan}8 &(32) choose {\color{cyan}pivot} \\ \hline
        \cellcolor{LightGreen}0 & \cellcolor{LightGreen}1 & \cellcolor{LightGreen}2 & \cellcolor{LightGreen}3 & \cellcolor{LightGreen}4 & \cellcolor{LightGreen}5 & \cellcolor{LightGreen}6 & \color{red}7 & \color{red}9 & \cellcolor{LightCyan}8 &(33) swap \(\color{red}9\) and \(\color{red}7\) \\ \hline
        \cellcolor{LightGreen}0 & \cellcolor{LightGreen}1 & \cellcolor{LightGreen}2 & \cellcolor{LightGreen}3 & \cellcolor{LightGreen}4 & \cellcolor{LightGreen}5 & \cellcolor{LightGreen}6 & 7 & \color{cyan}8 & \color{red}9 &(34) swap \(\color{red}9\) and {\color{cyan}pivot} \\ \hline
        \cellcolor{LightGreen}0 & \cellcolor{LightGreen}1 & \cellcolor{LightGreen}2 & \cellcolor{LightGreen}3 & \cellcolor{LightGreen}4 & \cellcolor{LightGreen}5 & \cellcolor{LightGreen}6 & 7 & \cellcolor{LightGreen}8 & 9 &(35) \(\color{green}8\) is in the correct place \\ \hhline{===========}
        \multicolumn{11}{ | c | }{partition the sequence into \((7)\) and \((9)\)} \\ \hhline{===========}
        \cellcolor{LightGreen}0 & \cellcolor{LightGreen}1 & \cellcolor{LightGreen}2 & \cellcolor{LightGreen}3 & \cellcolor{LightGreen}4 & \cellcolor{LightGreen}5 & \cellcolor{LightGreen}6 & \cellcolor{Amber}7 & \cellcolor{LightGreen}8 & \cellcolor{LightGrey}9 &(36) sort {\color{DarkOrange}left side} \\ \hline
        \cellcolor{LightGreen}0 & \cellcolor{LightGreen}1 & \cellcolor{LightGreen}2 & \cellcolor{LightGreen}3 & \cellcolor{LightGreen}4 & \cellcolor{LightGreen}5 & \cellcolor{LightGreen}6 & \cellcolor{LightGreen}7 & \cellcolor{LightGreen}8 & \cellcolor{LightGrey}9 &(37) \(\color{green}7\) is in the correct place \\ \hline
        \cellcolor{LightGreen}0 & \cellcolor{LightGreen}1 & \cellcolor{LightGreen}2 & \cellcolor{LightGreen}3 & \cellcolor{LightGreen}4 & \cellcolor{LightGreen}5 & \cellcolor{LightGreen}6 & \cellcolor{LightGreen}7 & \cellcolor{LightGreen}8 & \cellcolor{Amber}9 &(38) sort {\color{DarkOrange}left side} \\ \hline
        \cellcolor{LightGreen}0 & \cellcolor{LightGreen}1 & \cellcolor{LightGreen}2 & \cellcolor{LightGreen}3 & \cellcolor{LightGreen}4 & \cellcolor{LightGreen}5 & \cellcolor{LightGreen}6 & \cellcolor{LightGreen}7 & \cellcolor{LightGreen}8 & \cellcolor{LightGreen}9 &(39) \(\color{green}9\) is in the correct place \\ \hline
        \cellcolor{LightGreen}0 & \cellcolor{LightGreen}1 & \cellcolor{LightGreen}2 & \cellcolor{LightGreen}3 & \cellcolor{LightGreen}4 & \cellcolor{LightGreen}5 & \cellcolor{LightGreen}6 & \cellcolor{LightGreen}7 & \cellcolor{LightGreen}8 & \cellcolor{LightGreen}9 &(40) every thing is correctly sorted \\ \hline
        \caption{An example of quicksort applied to the sequence \((7, 1, 5, 4, 9, 2, 8, 3, 0, 6)\). Note that the table above is presented purely to illustrate the procedure of the algorithm and may not reflect one-to-one its implementation on a computer. For example, before swapping two numbers, the algorithm needs to compare each number leading up to that number to the pivot which was skipped in the table to improve readability. The numbers in the parentheses in the most right columns are also merely for referencing a specific row and do not correlate with the number of steps the algorithm needs to sort the given sequence.} \\
    \end{longtable}
\end{center}

We start with a sequence \((7, 1, 5, 4, 9, 2, 8, 3, 0, 6)\) which has ten distinct elements from \(0\) to \(9\). The far right element, \(6\), is chosen as the pivot (row 1). At the same time, define a counter \(i\) and set it to \(-1\). Then, start comparing each element from the left to right to the pivot. If the element is larger (or equal) than the pivot, nothing happens, but if it is smaller than the pivot, increment \(i\) by one and swap the number that was compared to the pivot with the number on \(i\)-th place of the sequence. For example, \(7 > 6\) hence nothing is changed, but \(1 < 6\) therefore, \(i\) is set to \(0\) and \(5\) is swapped with the number on the \(0\)th place which is \(7\) (row 2). The next number \(5\) is also smaller than the pivot \(6\), therefore, \(i\) is increment to \(1\) and \(5\) is swapped with the number on the first place which is again \(7\). This procedure is done for each number (compare rows 3 to 7). Finally, the pivot is swapped with the number on the \(i\)-th place. In our case, \(i\) is \(6\) at the end and the initial pivot \(6\) is correctly placed after the swap (see rows 8 and 9).
\\
After placing the initial pivot correctly, the sequence is partitioned into the left and the right side of the pivot which only contain numbers smaller or larger (or equal) than the pivot respectively i.e. the two partitions are \((1, 5, 4, 2, 3, 0)\) and \((9, 7, 8)\) with the first partition containing only numbers smaller than the pivot. Now, quicksort which is a recursive algorithm is applied to both partitions. For example, in the left partition, \(0\) is chosen as the pivot (row 11).
\\
There are few interesting points. On row 14, the first partition is empty, therefore nothing is sorted. Few rows after in row 22, we bluntly wrote that \(5\) is in the correct place, but we've skipped multiple steps before. In actuallity, because every number in the partition \((4, 2, 3, 5)\) are smaller (or equal) to the pivot \(5\) they are all swapped with themselves i.e. \(4\) is swapped with \(4\) and \(2\) is swapped with \(2\) and so on.