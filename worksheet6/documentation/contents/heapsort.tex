\section{Heapsort}
\subsection{Intuition and Description}
Heapsort is a sorting algorithm which introduces a data structure, a binary tree (the \textit{heap}). A heap is a nearly complete binary tree. For example, consider the list from previous examples

\begin{equation*}
    (7, 1, 5, 4, 9, 2, 8, 3, 0, 6) \text{.}
\end{equation*}

This list can be arranged into a heap simply by placing the first element of the list at the root, then placing the next two elements as its children and so on (see figure \ref{fig:heap}).
\begin{wrapfigure}{r}{0.5\textwidth}
    \begin{tikzpicture}[level distance=1cm,
        level 1/.style={sibling distance=3cm},
        level 2/.style={sibling distance=1.5cm},
        level 3/.style={sibling distance=1cm}]
        \node {7}
            child {
                node {1}
                child {
                    node {4}
                    child {
                        node {3}
                    }
                    child {
                        node {0}
                    }
                }
                child {
                    node {9}
                    child {
                        node {6}
                    }
                }
            }
            child {
                node {5}
                child {
                    node {2}
                }
                child {
                    node {8}
                }
            };
    \end{tikzpicture}
    \caption{the given list arranged into a heap}\label{fig:heap}
\end{wrapfigure}

In essence, the goal of heapsort is to first sort the heap into a \textit{max-heap} where each parent is larger than each of its children. Then, the root element which by necessity must be the largest element is removed from the heap and the element on the lowest branch (in figure \ref{fig:heap} where 6 currently is) is moved to the root (this procedure as a function is called \textit{build-max-heap}).

It turns out however, that not every recursion needs to check if the heap is a max-heap. After sorting the initial heap, one can assume that the heap is already sorted except for the root. The function which partially sorts the heap after the largest element was removed and the element of the lowest branch is moved to the top is called \textit{heapify}.\cite[p.~135]{bib:introductiontoalgorithms} 

As we did with quicksort, we present the pseudocode for heapsort in the following.

\begin{lstlisting}
def heapsort(_list):
    def heapify(_list, _n, _i):
        largest_element_index = _i

        LEFT_CHILD_INDEX = 2 * _i + 1
        if LEFT_CHILD_INDEX < _n:
            if (_list[LEFT_CHILD_INDEX] > 
                _list[largest_element_index]):
                largest_element_index = LEFT_CHILD_INDEX
    
        RIGHT_CHILD_INDEX = 2 * _i + 2
        IF RIGHT_CHILD_INDEX < _n:
            if (_list[RIGHT_CHILD_INDEX] >
                _list[largest_element_index]):
                largest_element_index = RIGHT_CHILD_INDEX
        
        if largest_element_index != _i:
            _list[_i], _list[largest_element_index] =
              _list[largest_element_index], _list[_i]

            heapify(_list, _n, largest_element_index)
            
    i = floor(len(_list) / 2) - 1

    while i >= 0:
        heapify(_list, len(_list), i)
        --i
    
    i = len(_list) - 1
    while i >= 0:
        _list[0], _list[i] = _list[0], _list[i]
        heapify(_list, i, 0)
        --i
\end{lstlisting}

\subsection{Worked Example}
As before, consider the list

\begin{equation*}
    (7, 1, 5, 6, 9, 2, 8, 3, 0, 6)\text{.}
\end{equation*}

We will sort this list using heapsort in the following (see figure \ref{fig:heapsortexample}). Due to space restrictions, steps between max heaps were condensed into one.

\textbf{Color Key}

\begin{itemize}
    \item the numbers swapped in the last step are in {\color{red}red} and {\color{green}green}, where {\color{green}green} indicates that the number was previously the root of the heap
    \item numbers {\color{gray}grayed} out are the numbers which are correctly sorted (removed from the heap)
\end{itemize}
\subsection{Complexity}

The initial build-max-heap takes time \(O(n)\) and heapify which takes time \(O(\ln(n))\) and is called \(n - 1\) times. Together, this means that the complexity of heapsort is \(O(n \ln(n))\).\cite[p.~136]{bib:introductiontoalgorithms}

Heapsort is not stable. Consider the following list \(2^{*}, 1, 2
)\). If heapsort is applied, this list is sorted to \(2, 1, 2^{*}\) thus the algorithm is unstable.

\begin{center}
%
% 1 to 10
%
\begin{figure}
    \caption{Worked example of heapsort.}\label{fig:heapsortexample}
    \begin{subfigure}[c]{0.5\textwidth}
    \begin{tikzpicture}[level distance=1cm,
        level 1/.style={sibling distance=3cm},
        level 2/.style={sibling distance=1.5cm},
        level 3/.style={sibling distance=1cm}]
        \node {7}
            child {
                node {1}
                child {
                    node {4}
                    child {
                        node {3}
                    }
                    child {
                        node {0}
                    }
                }
                child {
                    node {9}
                    child {
                        node {6}
                    }
                }
            }
            child {
                node {5}
                child {
                    node {2}
                }
                child {
                    node {8}
                }
            };
    \end{tikzpicture}
    \begin{tabular}{ | c | c | c | c | c | c | c | c | c | c | }
        \hline
        7 & 1 & 5 & 4 & 9 & 2 & 8 & 3 & 0 & 6 \\ \hline
    \end{tabular}
    \caption{(0) initial state}
    \end{subfigure}
    \begin{subfigure}[c]{0.5\textwidth}
    \begin{tikzpicture}[level distance=1cm,
        level 1/.style={sibling distance=3cm},
        level 2/.style={sibling distance=1.5cm},
        level 3/.style={sibling distance=1cm}]
        \node {\color{red}1}
            child {
                node {7}
                child {
                    node {4}
                    child {
                        node {3}
                    }
                    child {
                        node {0}
                    }
                }
                child {
                    node {6}
                    child {
                        node {\color{DarkGreen}9}
                    }
                }
            }
            child {
                node {8}
                child {
                    node {2}
                }
                child {
                    node {5}
                }
            };
    \end{tikzpicture}
    \begin{tabular}{ | c | c | c | c | c | c | c | c | c | c | }
        \hline
        \color{red}1 & 7 & 8 & 4 & 6 & 2 & 5 & 3 & 0 & \color{DarkGreen}9 \\ \hline
    \end{tabular}
    \caption{(1) create max heap; then swap {\color{red}1} and {\color{DarkGreen}9}}
    \end{subfigure}
    \begin{subfigure}[c]{0.5\textwidth}
    \begin{tikzpicture}[level distance=1cm,
        level 1/.style={sibling distance=3cm},
        level 2/.style={sibling distance=1.5cm},
        level 3/.style={sibling distance=1cm}]
        \node {\color{red}0}
            child {
                node {7}
                child {
                    node {4}
                    child {
                        node {3}
                    }
                    child {
                        node {\color{DarkGreen}8}
                    }
                }
                child {
                    node {6}
                    child {
                        [LightGrey]
                        node {\color{LightGrey}9}
                    }
                }
            }
            child {
                node {5}
                child {
                    node {2}
                }
                child {
                    node {1}
                }
            };
    \end{tikzpicture}
    \begin{tabular}{ | c | c | c | c | c | c | c | c | c | c | }
        \hline
        \color{red}0 & 7 & 5 & 4 & 6 & 2 & 1 & 3 & \color{DarkGreen}8 & \color{gray}9 \\ \hline
    \end{tabular}
    \caption{(2) create max heap; swap {\color{red}0} and {\color{DarkGreen}8}}
    \end{subfigure}
    \begin{subfigure}[c]{0.5\textwidth}
    \begin{tikzpicture}[level distance=1cm,
        level 1/.style={sibling distance=3cm},
        level 2/.style={sibling distance=1.5cm},
        level 3/.style={sibling distance=1cm}]
        \node {\color{red}3}
            child {
                node {6}
                child {
                    node {4}
                    child {
                        node {\color{DarkGreen}7}
                    }
                    child {
                        [LightGrey]
                        node {\color{LightGrey}8}
                    }
                }
                child {
                    node {0}
                    child {
                        [LightGrey]
                        node {\color{LightGrey}9}
                    }
                }
            }
            child {
                node {5}
                child {
                    node {2}
                }
                child {
                    node {1}
                }
            };
    \end{tikzpicture}
    \begin{tabular}{ | c | c | c | c | c | c | c | c | c | c | }
        \hline
        \color{red}3 & 6 & 5 & 4 & 0 & 2 & 1 & \color{DarkGreen}7 & \color{gray}8 & \color{gray}9 \\ \hline
    \end{tabular}
    \caption{(3) create max heap; swap {\color{red}3} and {\color{DarkGreen}7}}
    \end{subfigure}
    \begin{subfigure}[c]{0.5\textwidth}
    \begin{tikzpicture}[level distance=1cm,
        level 1/.style={sibling distance=3cm},
        level 2/.style={sibling distance=1.5cm},
        level 3/.style={sibling distance=1cm}]
        \node {\color{red}1}
            child {
                node {4}
                child {
                    node {3}
                    child {
                        [LightGrey]
                        node {\color{LightGrey}7}
                    }
                    child {
                        [LightGrey]
                        node {\color{LightGrey}8}
                    }
                }
                child {
                    node {0}
                    child {
                        [LightGrey]
                        node {\color{LightGrey}9}
                    }
                }
            }
            child {
                node {5}
                child {
                    node {2}
                }
                child {
                    node {\color{DarkGreen}6}
                }
            };
    \end{tikzpicture}
    \begin{tabular}{ | c | c | c | c | c | c | c | c | c | c | }
        \hline
        \color{red}1 & 4 & 5 & 3 & 0 & 2 & \color{DarkGreen}6 & \color{gray}7 & \color{gray}8 & \color{gray}9 \\ \hline
    \end{tabular}
    \caption{(4) create max heap; swap {\color{red}1} and {\color{DarkGreen}6}}
    \end{subfigure}
    \begin{subfigure}[c]{0.5\textwidth}
    \begin{tikzpicture}[level distance=1cm,
        level 1/.style={sibling distance=3cm},
        level 2/.style={sibling distance=1.5cm},
        level 3/.style={sibling distance=1cm}]
        \node {\color{red}1}
            child {
                node {4}
                child {
                    node {3}
                    child {
                        [LightGrey]
                        node {\color{LightGrey}7}
                    }
                    child {
                        [LightGrey]
                        node {\color{LightGrey}8}
                    }
                }
                child {
                    node {0}
                    child {
                        [LightGrey]
                        node {\color{LightGrey}9}
                    }
                }
            }
            child {
                node {2}
                child {
                    node {\color{DarkGreen}5}
                }
                child {
                    [LightGrey]
                    node {\color{LightGrey}6}
                }
            };
    \end{tikzpicture}
    \begin{tabular}{ | c | c | c | c | c | c | c | c | c | c | }
        \hline
        \color{red}1 & 4 & 2 & 3 & 0 & \color{DarkGreen}5 & \color{gray}6 & \color{gray}7 & \color{gray}8 & \color{gray}9 \\ \hline
    \end{tabular}
    \caption{(5) create max heap; swap {\color{red}1} and {\color{DarkGreen}5}}
    \end{subfigure}
        \begin{subfigure}[c]{0.5\textwidth}
        \begin{tikzpicture}[level distance=1cm,
            level 1/.style={sibling distance=3cm},
            level 2/.style={sibling distance=1.5cm},
            level 3/.style={sibling distance=1cm}]
            \node {\color{red}0}
                child {
                    node {3}
                    child {
                        node {1}
                        child {
                            [LightGrey]
                            node {\color{LightGrey}7}
                        }
                        child {
                            [LightGrey]
                            node {\color{LightGrey}8}
                        }
                    }
                    child {
                        node {\color{DarkGreen}4}
                        child {
                            [LightGrey]
                            node {\color{LightGrey}9}
                        }
                    }
                }
                child {
                    node {2}
                    child {
                        [LightGrey]
                        node {\color{LightGrey}5}
                    }
                    child {
                        [LightGrey]
                        node {\color{LightGrey}6}
                    }
                };
    \end{tikzpicture}
    \begin{tabular}{ | c | c | c | c | c | c | c | c | c | c | }
        \hline
        \color{red}0 & 3 & 2 & 1 & \color{DarkGreen}4 & \color{gray}5 & \color{gray}6 & \color{gray}7 & \color{gray}8 & \color{gray}9 \\ \hline
    \end{tabular}
    \caption{(6) create max heap; swap {\color{red}0} and {\color{DarkGreen}4}}
    \end{subfigure}
    \begin{subfigure}[c]{0.5\textwidth}
        \begin{tikzpicture}[level distance=1cm,
            level 1/.style={sibling distance=3cm},
            level 2/.style={sibling distance=1.5cm},
            level 3/.style={sibling distance=1cm}]
            \node {\color{red}0}
                child {
                    node {1}
                    child {
                        node {\color{DarkGreen}3}
                        child {
                            [LightGrey]
                            node {\color{LightGrey}7}
                        }
                        child {
                            [LightGrey]
                            node {\color{LightGrey}8}
                        }
                    }
                    child {
                        [LightGrey]
                        node {\color{LightGrey}4}
                        child {
                            [LightGrey]
                            node {\color{LightGrey}9}
                        }
                    }
                }
                child {
                    node {2}
                    child {
                        [LightGrey]
                        node {\color{LightGrey}5}
                    }
                    child {
                        [LightGrey]
                        node {\color{LightGrey}6}
                    }
                };
    \end{tikzpicture}
    \begin{tabular}{ | c | c | c | c | c | c | c | c | c | c | }
        \hline
        \color{red}0 & 1 & 2 & \color{DarkGreen}3 & \color{gray}4 & \color{gray}5 & \color{gray}6 & \color{gray}7 & \color{gray}8 & \color{gray}9 \\ \hline
    \end{tabular}
    \caption{(7) create max heap; swap {\color{red}0} and {\color{DarkGreen}3}}
    \end{subfigure}
\end{figure}
\begin{figure}
    \begin{subfigure}[c]{0.5\textwidth}
        \begin{tikzpicture}[level distance=1cm,
            level 1/.style={sibling distance=3cm},
            level 2/.style={sibling distance=1.5cm},
            level 3/.style={sibling distance=1cm}]
            \node {\color{red}0}
                child {
                    node {1}
                    child {
                        [LightGrey]
                        node {\color{LightGrey}3}
                        child {
                            [LightGrey]
                            node {\color{LightGrey}7}
                        }
                        child {
                            [LightGrey]
                            node {\color{LightGrey}8}
                        }
                    }
                    child {
                        [LightGrey]
                        node {\color{LightGrey}4}
                        child {
                            [LightGrey]
                            node {\color{LightGrey}9}
                        }
                    }
                }
                child {
                    node {\color{DarkGreen}2}
                    child {
                        [LightGrey]
                        node {\color{LightGrey}5}
                    }
                    child {
                        [LightGrey]
                        node {\color{LightGrey}6}
                    }
                };
    \end{tikzpicture}
    \begin{tabular}{ | c | c | c | c | c | c | c | c | c | c | }
        \hline
        \color{red}0 & 1 & \color{DarkGreen}2 & \color{gray}3 & \color{gray}4 & \color{gray}5 & \color{gray}6 & \color{gray}7 & \color{gray}8 & \color{gray}9 \\ \hline
    \end{tabular}
    \caption{(8) create max heap; swap {\color{red}0} and {\color{DarkGreen}2}}
    \end{subfigure}
    \begin{subfigure}[c]{0.5\textwidth}
        \begin{tikzpicture}[level distance=1cm,
            level 1/.style={sibling distance=3cm},
            level 2/.style={sibling distance=1.5cm},
            level 3/.style={sibling distance=1cm}]
            \node {\color{red}0}
                child {
                    node {\color{DarkGreen}1}
                    child {
                        [LightGrey]
                        node {\color{LightGrey}3}
                        child {
                            [LightGrey]
                            node {\color{LightGrey}7}
                        }
                        child {
                            [LightGrey]
                            node {\color{LightGrey}8}
                        }
                    }
                    child {
                        [LightGrey]
                        node {\color{LightGrey}4}
                        child {
                            [LightGrey]
                            node {\color{LightGrey}9}
                        }
                    }
                }
                child {
                    [LightGrey]
                    node {\color{LightGrey}2}
                    child {
                        [LightGrey]
                        node {\color{LightGrey}5}
                    }
                    child {
                        [LightGrey]
                        node {\color{LightGrey}6}
                    }
                };
    \end{tikzpicture}
    \begin{tabular}{ | c | c | c | c | c | c | c | c | c | c | }
        \hline
        \color{red}0 & \color{DarkGreen}1 & \color{gray}2 & \color{gray}3 & \color{gray}4 & \color{gray}5 & \color{gray}6 & \color{gray}7 & \color{gray}8 & \color{gray}9 \\ \hline
    \end{tabular}
    \caption{(9) create max heap; swap {\color{red}0} and {\color{DarkGreen}1}}
    \end{subfigure}
    \begin{subfigure}[c]{0.5\textwidth}
        \begin{tikzpicture}[level distance=1cm,
            level 1/.style={sibling distance=3cm},
            level 2/.style={sibling distance=1.5cm},
            level 3/.style={sibling distance=1cm}]
            \node {\color{DarkGreen}0}
                child {
                    [LightGrey]
                    node {\color{LightGrey}1}
                    child {
                        [LightGrey]
                        node {\color{LightGrey}3}
                        child {
                            [LightGrey]
                            node {\color{LightGrey}7}
                        }
                        child {
                            [LightGrey]
                            node {\color{LightGrey}8}
                        }
                    }
                    child {
                        [LightGrey]
                        node {\color{LightGrey}4}
                        child {
                            [LightGrey]
                            node {\color{LightGrey}9}
                        }
                    }
                }
                child {
                    [LightGrey]
                    node {\color{LightGrey}2}
                    child {
                        [LightGrey]
                        node {\color{LightGrey}5}
                    }
                    child {
                        [LightGrey]
                        node {\color{LightGrey}6}
                    }
                };
    \end{tikzpicture}
    \begin{tabular}{ | c | c | c | c | c | c | c | c | c | c | }
        \hline
        \color{DarkGreen}0 & \color{gray}1 & \color{gray}2 & \color{gray}3 & \color{gray}4 & \color{gray}5 & \color{gray}6 & \color{gray}7 & \color{gray}8 & \color{gray}9 \\ \hline
    \end{tabular}
    \caption{(10) create max heap; swap {\color{DarkGreen}0} and {\color{DarkGreen}0}}
    \end{subfigure}
    \begin{subfigure}[c]{0.5\textwidth}
        \begin{tikzpicture}[level distance=1cm,
            level 1/.style={sibling distance=3cm},
            level 2/.style={sibling distance=1.5cm},
            level 3/.style={sibling distance=1cm}]
            \node {0}
                child {
                    node {1}
                    child {
                        node {3}
                        child {
                            node {7}
                        }
                        child {
                            node {8}
                        }
                    }
                    child {
                        node {4}
                        child {
                            node {9}
                        }
                    }
                }
                child {
                    node {2}
                    child {
                        node {5}
                    }
                    child {
                        node {6}
                    }
                };
    \end{tikzpicture}
    \begin{tabular}{ | c | c | c | c | c | c | c | c | c | c | }
        \hline
        0 & 1 & 2 & 3 & 4 & 5 & 6 & 7 & 8 & 9 \\ \hline
    \end{tabular}
    \caption{the heap is correctly sorted}
    \end{subfigure}
\end{figure}
\end{center}